\section{Type System}

The goal of the type system is to minimise programming errors, but on the other hand not being too restrictive so the expressiveness of the language is reduced. 

The Language Described in this Report is a statically typed language. Statically typed languages have the following nice properties, that do not apply to dynamically typed languages.

\begin{itemize}
  \item Type errors are presented at the soonest possible level: compile-time. This makes it impossible to have a program stop unexpectedly because of a type error at run-time
  \item Because of the fact that types are checked at compile time, no overhead is imposed to the compiled program at run-time
\end{itemize}

Of course, dynamically typed languages have its uses. One can argue that prototyping a program is done faster in a dynamically typed language. The trade-off is usually expressiveness for safety and performance.

In order to reduce the mental overhead and the number of explicit type declarations to write in a language, type inference can be implemented in the compiler. Having almost all types inferred by the compiler makes the language have more of the same feel as a dynamically typed language, in that there is less mental overhead for the user. Using a statically typed language with type inference brings all the beneficial properties of statically typed language, combined with a \enquote{lighter} feel.

\subsection{Strictness}

The type system defined in this report can be described as a \emph{strong} type system, as opposed to a \emph{weak} type system. No formal definitions of such categorisations of type systems exist, so this report will use the following understanding of \emph{weak} and \emph{strong} type systems. A type system goes from \emph{weak} to \emph{strong} as the number of undefined behaviour, unpredictable behaviour or implicit conversions between types approach zero.

Using the above definition the type system can be categorised as \emph{strong}, as only one implicit conversion is made between values of type \emph{int} to values of type \emph{real}, whenever number precision can be preserved. Other than that, the type system should be predictable.

\subsection{Higher Kinded Types}
\label{sub:higher_kinded_types}

Some programming languages allow the notion of higher kinded types, i.e. types of types. A kind is a type of a type constructor. One can describe a kind as one level up from a type. Kinds are often notated using \emph{*}. For example the kind of the type \emph{int} is \emph{*}. The kind of the function type \emph{int -> int} is of kind \emph{* -> *}.

Type systems with support for higher kinded types allow for even more guarantees that can be made about the program at compile-time.

An example of a program snippet declaring symbol \emph{x} with the kind \emph{*} and assigning it to \emph{int} can be seen below.

\begin{verbatim}
  let x:* := int
\end{verbatim}

This assignment is perfectly legal syntactically and semantically. Consider however the example below, in which a variable symbol initialisation takes place.

\begin{verbatim}
  var y:* := int
  var z:y := 4
  y := bool
  z := 5
\end{verbatim}

In this example, the mutable binding to y is initially bound to \emph{int}. The binding \emph{z} has the type of \emph{y} i.e. \emph{int}. On line 3, the type of \emph{y} is reassigned to \emph{bool}. On line 4, problems occur when \emph{z} is trying to be reassigned to \emph{5} which is of type \emph{int} and therefor not the correct type of \emph{bool}.

On solution to avoiding this problem, is to disallow mutable symbol bindings which is a kind. This would allow \emph{let y:* := int}, but disallow \emph{var y:* := int}. Another solution is to not include higher kinded types in the type system.

To avoid the confusion of the users of the language not knowing when it is allowed to have mutable bindings, the type system for The Language Described in this Report does not support higher kinded types.

%%%%%%%%%%%%%%%%%%%%%%%%%%%%%%%%%%%%%%%%%%%%%%%%%%%%%%%%%%%%%%%%% gammelt afsnit

% It has been decided to design the language in a manner that insures that all type checking can be done at compile time. To insures this; all identifiers must have explicit type declaration.
% A meta type is in this report defined as a type which can be an instance of any other type in the language. %Expand on this
% The functionally which spawns a new actor needs to know of which type the actor should be of, in this case a meta type would be useful to describe the actual type of the actor.
% So to imagine a meta type in this language, it would look something like this: %Bad language and not very academic

% \begin{verbatim}
%   let x:Type := Int
% \end{verbatim}

% In fact $typeX$ is just a placeholder for a type, the following expression should evaluate to int and the declaration of $y$ would be fine. %Fine? Correct?
% But notice that type is declared using the var keyword which makes an variable mutable. %Rephrase
% If $typeX$ is mutable it can change state doing run-time, and in this case the type of $y$ can not be determined at compile time, which disagrees with the premise mentioned before. % What why?

% \begin{verbatim}
%   var typeX:Type := Int
%   let y:typeX := 4
% \end{verbatim}

% Different solutions for this problem have been considered, one way of doing it is to disallow mutable meta type. This solution insures that the meta type can't change at run-time, and there for the type can be determined at compile time. Another way of achieving compile time checking, is to disallow a variable to be type declared using a meta type. Due to a goal of having decent orthogonality in the language, it has been decided to not implement any kind of meta type. This solution ensures that there will be no confusion about how a meta type can be declared and no to be declared.
% In \cref{sub:constructionOfAnActor} it is described, how it is possible to spawn an actor without the use of any meta type.


\subsection{Type Rules}

% macros for symbols:
\newcommand{\Tpot}{^\wedge{}}
\newcommand{\Tint}{\text{int}}
\newcommand{\Treal}{\text{real}}
\newcommand{\Tbool}{\text{bool}}
\newcommand{\Tchar}{\text{char}}
\newcommand{\Tvoid}{\text{void}}
\newcommand{\Taop}{\text{AOP}}
\newcommand{\Tbaop}{\text{BAOP}}
\newcommand{\Tbop}{\text{BOP}}
\newcommand{\Tconct}{\text{CONCT}}
\newcommand{\Tpt}{\text{PT}}
\newcommand{\Tenv}{E \vdash}

\subsubsection{Common Notation}
The following variables are used when any elements of the set can be used. For example the variable $AOP$, Arithmetic OPeration, can be a $+$.
\paragraph{Operators}
\begin{align*}
&\Taop = \left\{ {+, -, *, /, \%, \; \Tpot \;} \right\}
\\
&\Tbaop = \left\{ {<, <=, >, >=} \right\}
\\
&\Tbop = \left\{ {\text{AND, NAND, OR, NOR, XOR}} \right\}
\end{align*}

\paragraph{Types}
\begin{align*}
&\Tpt = \left\{ {\Tint, \Treal, \Tbool, \Tchar} \right\}
\\
&\text{TUPLE} = (\Tconct, \Tconct^+)
\\            
&\Tconct = \left\{ { \left[ \Tconct \right], \Tpt, \text{TUPLE}, \text{struct<TUPLE>}} \right\}
\end{align*}

\subsubsection{Arithmetic Expressions}
\begin{align*}
&\inference[$\text{EXPR}_{\Tint,\Tint}$]{\Tenv e_1  : \Tint & 
                       \Tenv e_2 : \Tint}
                    {\Tenv e_1 \mathbin{\Taop} e_2 : \Tint}
%%%%%%%%%%%%%%%%%%%%%%%%%%%%%%%%%%%%%%%%%%%%%%%%%%%%%%%%%%%%%%%%%%%%%%%%%%%%%%%%%%%%%%%%
\\\\
&\inference[$\text{EXPR}_{\Treal,\Treal}$]{\Tenv e_1 : \Treal & 
                       \Tenv e_2 : \Treal}
                    {\Tenv e_1 \mathbin{\Taop} e_2 : \Treal}
%%%%%%%%%%%%%%%%%%%%%%%%%%%%%%%%%%%%%%%%%%%%%%%%%%%%%%%%%%%%%%%%%%%%%%%%%%%%%%%%%%%%%%%%
\\\\
&\inference[$\text{ROOT}_{\Tint,\Tint}$]{\Tenv e_1 : \Tint &
                       \Tenv e_2 : \Tint}
                    {\Tenv e_1 \mathbin{\#} e_2 : \Treal}
%%%%%%%%%%%%%%%%%%%%%%%%%%%%%%%%%%%%%%%%%%%%%%%%%%%%%%%%%%%%%%%%%%%%%%%%%%%%%%%%%%%%%%%%
\\\\
&\inference[$\text{ROOT}_{\Treal,\Treal}$]{\Tenv e_1 : \Treal &
                       \Tenv e_2 : \Treal}
                    {\Tenv e_1 \mathbin{\#} e_2 : \Treal}
\end{align*}
\sinote{root med ints -> int er endnu ikke komplet. Skal det altid give en real?}

\subsubsection{Boolean Expressions}
\begin{align*}
&\inference[$\text{BOOL}_{\Tbaop-\Tint,\Tint}$]{\Tenv e_1 : \Tint & 
                       \Tenv e_2 : \Tint}
                    {\Tenv e_1 \mathbin{\Tbaop} e_2 : \Tbool}
%%%%%%%%%%%%%%%%%%%%%%%%%%%%%%%%%%%%%%%%%%%%%%%%%%%%%%%%%%%%%%%%%%%%%%%%%%%%%%%%%%%%%%%%
\\\\
&\inference[$\text{BOOL}_{\Tbaop-\Treal,\Treal}$]{\Tenv e_1 : \Treal &
                       \Tenv e_2 : \Treal}
                    {\Tenv e_1 \mathbin{\Tbaop} e_2 : \Tbool}
%%%%%%%%%%%%%%%%%%%%%%%%%%%%%%%%%%%%%%%%%%%%%%%%%%%%%%%%%%%%%%%%%%%%%%%%%%%%%%%%%%%%%%%%
\\\\
&\inference[$\text{BOOL}_{EQUALS}$]{\Tenv e_1 : \Tconct &
                       \Tenv e_2 : \Tconct}
                    {\Tenv e_1 = e_2 : \Tbool}
%%%%%%%%%%%%%%%%%%%%%%%%%%%%%%%%%%%%%%%%%%%%%%%%%%%%%%%%%%%%%%%%%%%%%%%%%%%%%%%%%%%%%%%%
\\\\
&\inference[$\text{BOOL}_{NEQUALS}$]{\Tenv e_1 : \Tconct &
                       \Tenv e_2 : \Tconct}
                    {\Tenv e_1 \mathbin{\text{!=}} e_2 : \Tbool}
%%%%%%%%%%%%%%%%%%%%%%%%%%%%%%%%%%%%%%%%%%%%%%%%%%%%%%%%%%%%%%%%%%%%%%%%%%%%%%%%%%%%%%%%
\\\\
&\inference[$\text{BOOL}_{BBOP}$]{\Tenv e_1 : \Tbool &
                       \Tenv e_2 : \Tbool}
                    {\Tenv e_1 \mathbin{\Tbop} e_2 : \Tbool}
\end{align*}
\subsubsection{Conditional Expressions}
\begin{align*}
\intertext{The conditional body of a if-statement must be of type bool. The body can be of any type defined in $\Tconct$. The type of the if-statement is void.}
&\inference[$\text{IF}$]{\Tenv b : \Tbool &
                  \Tenv e : \Tconct}
                 {\Tenv \mathbin{\text{if}} \; (b) \; \{e\}: \Tvoid}
%%%%%%%%%%%%%%%%%%%%%%%%%%%%%%%%%%%%%%%%%%%%%%%%%%%%%%%%%%%%%%%%%%%%%%%%%%%%%%%%%%%%%%%%
\intertext{The conditional body of a if-else-statement must be of type bool. The two bodies can be of any type defined in $\Tconct$. The type of the if-else-statement is void.}
&\inference[$\text{IF-ELSE}$]{\Tenv b : \Tbool &
                  \Tenv e_1 : \Tconct &
                  \Tenv e_2 : \Tconct'}
                 {\Tenv \mathbin{\text{if}} \; (b) \; \{e_1\} \mathbin{\text{else}} \{e_2\}: \Tvoid}
\end{align*}

\subsubsection{Loop Expressions}
\begin{align*}
\intertext{The conditional body of the while construct must be of type bool. The body of the while-loop can be of any type defined in $\Tconct$}
&\inference[WHILE]{\Tenv b : \Tbool &
                  \Tenv e : \Tpt}
                 {\Tenv \mathbin{\text{while}} \; (b) \; {e}: \Tvoid}
%%%%%%%%%%%%%%%%%%%%%%%%%%%%%%%%%%%%%%%%%%%%%%%%%%%%%%%%%%%%%%%%%%%%%%%%%%%%%%%%%%%%%%%%
\intertext{For loops can only work on lists. The counter variable is the type of a element in the list being iterated.}
&\inference[FOR]{\Tenv l : [\Tpt]}
                 {\Tenv \mathbin{\text{for}} \; \mathbin{\text{x}} \; \mathbin{\text{in}} \; {l} \; : \Tvoid },	 \Tenv x : \Tpt
%%%%%%%%%%%%%%%%%%%%%%%%%%%%%%%%%%%%%%%%%%%%%%%%%%%%%%%%%%%%%%%%%%%%%%%%%%%%%%%%%%%%%%%%
\\\\                                  
&\inference[FOR]{\Tenv l : [\Tpt]} 
                 {\Tenv \mathbin{\text{for}} (x,y) \mathbin{\text{in}} \; {l} \; : \Tvoid } , \Tenv x : \Tpt, \Tenv y : \Tint
\end{align*}

\subsubsection{Match statements}
\begin{align*}
\intertext{When matching on an expression of type $\Tconct$ each match case must be of the same type as the expression. There can be multiple match cases as denoted by the plus. The body of each match case can be any type of $\Tconct'$ different from the type of the expression being matched on. The whole match statement has type void.}
&\inference[MATCH]{\Tenv e_1 : \Tconct &
                   \Tenv e_2 : \Tconct &
                   \Tenv S : \Tconct'}
                 {\Tenv \mathbin{\text{match}} \; e_1 \; {(e_2 -> S)^+}: \Tvoid}
\end{align*}
