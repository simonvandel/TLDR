\section{Statements}\label{sec:statements}
Statements are in TLDR defined as constructs that have the possibility to change the state of the program. The state of the program is defined as the symbol by which values are associated. We call this the environment and it is formally described as:
\begin{center}
$e = \text{Symbols} \rightharpoonup \text{Stm} \times \text{Symbols}$
\end{center}
This is a partial function that maps symbols to statements and other statements. These statements are the statements that are run when the symbol is invoked. If these maps to an expression the invocation takes the value of that invocation in the current environment. This can be any expression and as explained the symbol evaluates to whatever the expression maps to. Note that a block is also an expression, see \cref{subsubsec:invocation}. If the statements evaluates only to a statement the symbol is of type \enquote{unit} and it has no value. Note that the statements can still update the state of the program during the invocation. 

If the symbol takes input arguments these are placed inside the \enquote{Symbols} on the right hand side of the arrow. In an invocation these symbols will be mapped to the values given as input parameters.

\subsection{Value Assignment}
\label{subsec:value_assignment}

A value can be assigned to a symbol in either a constant binding or a variable binding.

In traditional mathematical notation, the \enquote{=} symbol is also sometimes used to let certain symbols represent a more complex meaning, in order to simplify something, such as an equation or a function. When used like this, often mathematicians put the word \enquote{let} in front of a statement to denote that it is a definition. We wanted to follow this construct as well letting immutable assignment be denoted in this fashion, since they are comparable to definitions. 

But since we wanted assignments, be it mutable or immutable, to have a similarities, we chose \enquote{:=} for all assignments. This concept is less known in traditional mathematics, but is widely used in computational science. In the historically significant languages Fortran and C, the \enquote{=} symbol, was used for this. However, since we wish to keep that symbol closer to its original meaning, we needed something else. \enquote{:=} was chosen, since it is a known symbol from other languages. The asymmetry of the \enquote{:=} symbol also illustrates that it matters which side of the symbol a variable is on, as opposed to the \enquote{=} symbol.

The grammar for value assignments ended up following this grammar:

\kanote{here goes grandmar}

as can be seen, values can either be an immutable constant, or a muteable variable.

A constant binding can never have its value changed. For example, if \enquote{a} is bound to \enquote{5}, \enquote{a} can never refer to another value than \enquote{5} in the same lexical scope. 

The syntax for constant value assignment is as follows.
\begin{verbatim}
  let <symbolName> : <type> := <value>
\end{verbatim}
A concrete example:

\begin{verbatim}
  let x : int := 2
\end{verbatim}

A variable binding can always change the value it refers to. For example, if \enquote{b} is bound to \enquote{2}, it is perfectly possible to later in the source code refer to {10}.

The syntax for variable value assignment is as follows.

\begin{verbatim}
  var <symbolName> : <type> := <value>

// a later reassignment
  <symbolName> := <value>
\end{verbatim}
A concrete example:

\begin{verbatim}
  var a : int := 2
  
// a later reassignment
  a := 2
\end{verbatim}

The semantics for declaring, initializing and assigning variables is as follows:
\kanote{semantik: decls er udkommmenteret. Hvorfor? er det ikke korrekt?}

\begin{align*}
%&\inference[$\text{DECL}$]{}
%                         {\Braket{\Tlet x := a,e,st} \Rightarrow_S \Braket{x := a, e',st}}
%												{, l = e(next), e' = e[x\mapsto l, next\mapsto new(l)]}
%\\\\
&\inference[$\text{INIT}_{SYM}$]{}
                         {\Braket{\Tlet \Tx := \{S\};,e} \Rightarrow_S e'}
												 {, e' = e[\Tx \mapsto \Braket{S,\epsilon}]}
\\\\
&\inference[$\text{INIT}$]{}
                         {\Braket{\Tvar \Tx := S,e} \Rightarrow_S e'}
												 {, e' = e[\Tx \mapsto \Braket{S,\epsilon}]}
\\\\
&\inference[ASS]{}
                 {\Braket{\Tx := S,e} \Rightarrow_S e'}
								 {, e' = e[\Tx \mapsto \Braket{S,\epsilon}]}
\end{align*}
\subsection{Functions}
\label{subsec:functions}

Functions can be declared in two ways. By separating the type signature and the function body or by combining the signature and the function body.

\paragraph{Separated function declaration}

The syntax for the separated function declaration is as follows. The type signature and the body must be declared in the same lexical scope.

\begin{verbatim}
  <funcName> : <typeSignature>;
  let <funcName>(<parameterList>) := {<body>};
\end{verbatim}

A concrete example:

\begin{verbatim}
  plus : int -> int -> Int;
  let plus(x, y) := {x + y};
\end{verbatim}


\paragraph{Combined function declaration}

The syntax for the combined function declaration is as follows.

\begin{verbatim}
  let <funcName>(<parameterList>) : <typeSignature> := {<body>};
\end{verbatim}

A concrete example:

\begin{verbatim}
  let plus(x, y) : Int -> Int -> Int := {x + y};
\end{verbatim}

%mainfile: ../../master.tex
\subsection{Structures}
\label{subsec:structs}

In TLDR there are three ways to do encapsulation. Actors, Tuples and structs. Structs are unique by being fully accessible within the scope, and having named fields. Structs are especially useful in TLDR for creating messages.

\subsubsection{Defining Structures}
\label{sec:defStructures}

\subsubsection{Syntax}

Structures are defined by using the \enquote{struct} keyword. The grammar for declaring structs are as follows:

\begin{grammar}
  <Struct> ::= 'struct' <Identifier> ':= \{' <TypeDecls> '\}'
\end{grammar}

And a concrete example:

\begin{lstlisting}[style=TLDR]
  struct Person := {Name:[char]; Age:int}
\end{lstlisting}

\subsubsection{Semantics}

having these semantics:

\begin{align*}
\intertext{In the case that we have multiple S of assignments, we can rewrite the T to now map to new s' that includes x, in the new st' environment and the rest of declaration statements}
&\inference[$\text{STRUCT}$]{}
                            {\Braket{\Tstruct T := \{x:T';S\}, env_s, st} \Rightarrow_S \Braket{\Tstruct T := \{S\},env_s,st'}}
\\
&{\Twhere st' = st[T \mapsto s'],st(T) = s,s' = s \cup x]}
\\\\
\intertext{In the case that we have multiple S of assignments, we can rewrite the T to now map to new s' that includes x, in the new st' environment}
&\inference[$\text{STRUCT}$]{}
                            {\Braket{\Tstruct T := \{x:T\}, env_s, st} \Rightarrow_S \Braket{env_s,st'}}
\\
&{\Twhere st' = st[T \mapsto s'],st(T) = s,s' = s \cup x]}
\end{align*}

\begin{align*}
&\inference[$\text{STRUCT}$]{}
                            {\Braket{\Tstruct T_1 := \{x:T_2;S\}, env_s} \Rightarrow_S \Braket{\Tstruct T_1 := \{S\},env_s'}}
\\
&{\Twhere env_s' = env_s[T \mapsto \Braket{\epsilon,s'}],env_s(T) = s,s' = s \cup x]}
\\\\
&\inference[$\text{STRUCT}$]{}
                            {\Braket{\Tlet \; x:T := (S),env_s} \Rightarrow_S \Braket{x:T := (S),env_s'}}
\\
&{env_s' = env_s[x \mapsto (S,s)],env_s(T) = (\epsilon,s)]}
\\\\
&\inference[$\text{STRUCT}$]{}
                            {\Braket{s:T := (f := x;S),env_s} \Rightarrow_S \Braket{s.f := x;s:T := (S),env_s}}
\end{align*}

\subsubsection{Type Rules}

\begin{align*}
\intertext{The elements can be of any type defined in $\Tt$}
&\inference[STRUCT]{E[s \mapsto (e_1:\Tt_1;e_2:\Tt_2;...;e_n:\Tt_n) \rightarrow ok]\vdash S : ok & }
                 {\Tenv \mathbin{\text{struct s}} := \{e_1:\Tt_1;e_2:\Tt_2;...;e_n:\Tt_n\}; S: ok}
\end{align*}



\subsubsection{Initialising Structures}
\label{sec:initStructures}

Structs can be initialised and assigned to symbols using either a constant assignment or a variable assignment. Structs initialised as a constant assignment cannot change any of the fields of the structs; the struct is immutable. Structs initialised as a variable assignment can change all of its fields at any time; the struct is mutable.

\subsubsection{Syntax}

The syntax for initialising a struct is as follow.

\begin{grammar}
<StructLiteral> ::= '(' (<Reassignment>';')* ')' (':' <Identifier>)?
\end{grammar}


With concrete examples for immutables:

\begin{verbatim}
  let Alice:Person := (Name := "Alice"; Age := 20);
\end{verbatim}

And for mutable struct is as follow.

\begin{verbatim}
  var Alice:Person := (Name := "Alice"; Age := 20);
\end{verbatim}

And for usage in lists.

\begin{verbatim}
  [(Name := "Alice"; Age := 20):Person];
\end{verbatim}

\subsubsection{Access to Structure Fields}
\label{sec:accessStructFields}

Fields can be access using the following syntax.

\begin{verbatim}
  Alice.Name; // "Alice"
\end{verbatim}

Structs declared as mutable can have fields reassigned using the following syntax.

\begin{verbatim}
  Alice.Name; // "Alice"
  Alice.Name := "Bob";
  Alice.Name; // "Bob";
\end{verbatim}
 
\subsubsection{Semantics}

\begin{align*}
\intertext{In the case that we have multiple S of assignments, we can rewrite the s to now having an accessor that maps to value of the first assignment and the rest of pending assignments in s}
&\inference[$\text{STRUCT}$]{}
                            {\Braket{\Tlet \; s:T := (f := x;S),env_s,st} \Rightarrow_S \Braket{s.f := x;\Tlet \; s:T := (S),env_s,st}}
\\\\
\intertext{In the case that we have only one assignment, we can rewrite the s to now having an accessor that maps to value of the assignment}
&\inference[$\text{STRUCT}$]{}
                            {\Braket{\Tlet \; s:T := (f := x),env_s,st} \Rightarrow_S \Braket{s.f := x,env_s,st}}
%\\\\
%&\inference[$\text{STRUCT}$]{}
%                            {\Braket{(f := x;S),env_s,st} \Rightarrow_S \Braket{s.f := x;\Tlet \; s:T := (S),env_s,st}}
%\\\\
%&\inference[$\text{STRUCT}$]{}
%                            {\Braket{(f := x):T,env_s,st} \Rightarrow_S \Braket{s.f := x,env_s,st}}
\end{align*}

\subsubsection{Type Rules}

Each e of type t is matching the declared struct types, evaluates to the type of the declared struct.

\begin{align*}
&\inference[STRUCTLITERAL]{\Tenv (e_1 : \Tt_1;e_2 : \Tt_2;...;e_n : \Tt_n):\Tt'}
                 {\Tenv \mathbin{\text{(}} e_1; e_2;...;e_n\mathbin{\text{)}}:\Tt': \Tt'}
\end{align*}

\subsubsection{For-loop Statements}
\label{subsec:forLoopStatements}

There are two for-loop statements defined in the language. A for-in loop and a variation of the for-in loop where both the index and the element of the collection being looped is provided.

\paragraph{For-in loop}
\label{sec:forInLoop}

The for-in loop statement has the following syntax.

\begin{verbatim}
  for <element> in <collection> {<loopBody>}
\end{verbatim}

A concrete example:

\begin{verbatim}
  for i:int in [0..10]:[int] { /* Do stuff with i elements */ }
\end{verbatim}

\paragraph{For-in loop with index}
\label{sec:forInLoopIndex}

A variation of the for-in loop that also provides the index of the current element, has the following syntax.

\begin{verbatim}
  for (<index>, <element>) in <collection> {<loopBody>}
\end{verbatim}

A concrete example:

\begin{verbatim}
  for (index:int, elem:int) in [0..10]:[int] { 
    /* Do stuff with i elements and index */ 
  }
\end{verbatim}

%mainfile: ../master.tex
\subsection{While-loop}
\label{subsec:whileLoopStatements}

The while loop statement runs a block of statements until a boolean expression provided returns false.

\subsubsection{Syntax}

\begin{grammar}
<While> ::= 'while' <Expression> <Block>
\end{grammar}

A concrete example:

\begin{verbatim}
  var i:int := 0;
  while (i < 10) {i := i + 1}
\end{verbatim}

\subsubsection{Semantics}

\begin{align*}
\intertext{In the case that the boolean expression $b$ is true, the rule for composition of statements is used. First the statement $S$ is evaluated and then the while statement is run again. The statement $S$ may have changed $b$.}
&\inference[$\text{WHILE}_\top$]{env_S \vdash b \Rightarrow_B \top}
                       {\Braket{\Twhile(b)\{S\},env_s} \Rightarrow_S \Braket{\{S\}; \Twhile (b)\{S\},env_S}}
\\\\
\intertext{In the case that the boolean expression $b$ is false, the while statement has ended, and can simply be rewritten to the environment $env_S$}
&\inference[$\text{WHILE}_\bot$]{env_S \vdash b \Rightarrow_B \bot}
                       {\Braket{\Twhile(b)\{S\},env_S} \Rightarrow_S env_S}
\end{align*}

\subsubsection{Type Rules}

\begin{align*}
\intertext{The conditional body of the while construct must be of type bool. The body of the while-loop can be of any type defined in $\Tt$}
&\inference[WHILE]{\Tenv b : \Tbool &
                  \Tenv e : \Tt}
                 {\Tenv \mathbin{\text{while}} \; (b) \; {e}: ok}
\end{align*}

%mainfile: ../master.tex
\subsection{If-statements}
\label{subsec:ifStatements}

If-statements can be written either as a if-else statement or just as an if-statement. In an if-else statement, the body just after the condition is evaluated if the condition evaluates to the boolean value \emph{true}. If the condition evaluates to \emph{false}, the body after the else keyword is evaluated. In an if-statement, the body after the condition is run, if the condition has the value \emph{true}. If the condition has value \emph{false}, nothing is evaluated.

\subsubsection{Syntax}

\begin{grammar}
<If> ::= 'if' <Expression> <Block>

<IfElse> ::= 'if' <Expression> <Block> 'else' <Block>
\end{grammar}

A concrete example:

\begin{verbatim}
  // if-then-else statement
  if (2 + 2 = 4) {"math works!"}
  else {"something is wrong here!"}

  // if-statement
  if (remainingTime < 10) {initiateCountdown()}
\end{verbatim}

\subsubsection{Semantics}

\begin{align*}
\intertext{In the case that the condition $b$ is true, the rule for the if-statement can simply be rewritten to the statement $S$.}
&\inference[$\text{IF}_\top$]{}
                      {\Braket{if(b)\{S\},sEnv} \Rightarrow_S \Braket{\{S\},sEnv}}
                      {,b \Rightarrow_B \top}
\\\\
\intertext{In the case that the condition $b$ is false, nothing is evaluated, and the if-statement is rewritten to the symbol environment $sEnv$.}
&\inference[$\text{IF}_\bot$]{}
                      {\Braket{if(b)\{S\},sEnv} \Rightarrow_S sEnv}
                      {,b \Rightarrow_B \bot}
\\\\
\intertext{In the case that the condition $b$ is true, the rule for the if-else statement is rewritten to $S_1$.}
&\inference[$\text{IF-ELSE}_\top$]{}
                      {\Braket{if(b)\{S_1\}else\{S_2\},e} \Rightarrow_S \Braket{\{S_1\},e}}
                      {,b \Rightarrow_B \top}
\\\\
\intertext{In the case that the condition $b$ is false, the rule can be rewritten to $S_2$.}
&\inference[$\text{IF-ELSE}_\bot$]{}
                      {\Braket{if(b)\{S_1\}else\{S_2\},e} \Rightarrow_S \Braket{\{S_2\},e}}
                      {,b \Rightarrow_B \bot}
\end{align*}

\subsubsection{Type Rules}

\begin{align*}
\intertext{The conditional body of a if-statement must be of type bool. The body can be of any type defined in $\Tt$. The type of the if-statement is well-typed.}
&\inference[$\text{IF}$]{\Tenv b : \Tbool &
                  \Tenv e : \Tt}
                 {\Tenv \mathbin{\text{if}} \; (b) \; \{e\}: ok}
%%%%%%%%%%%%%%%%%%%%%%%%%%%%%%%%%%%%%%%%%%%%%%%%%%%%%%%%%%%%%%%%%%%%%%%%%%%%%%%%%%%%%%%%
\intertext{The conditional body of a if-else-statement must be of type bool. The two bodies can be of any type defined in $\Tt$. The type of the if-else-statement is well-typed.}
&\inference[$\text{IF-ELSE}$]{\Tenv b : \Tbool &
                  \Tenv e_1 : \Tt &
                  \Tenv e_2 : \Tt}
                 {\Tenv \mathbin{\text{if}} \; (b) \; \{e_1\} \mathbin{\text{else}} \{e_2\}: ok}
\end{align*}

\subsection{Match Statements}
\label{subsec:matchStatements}

Match statements can be seen as syntactical sugar for multiple chained if-statements. The syntax is as follows.

\begin{verbatim}
  match <whatToMatchOn> {
    <case1> -> <actionOnCase1>
    <case2> -> <actionOnCase2>
    ...
    <caseN> -> <actionOnCaseN>
  }
\end{verbatim}

A concrete example:

\begin{verbatim}
  match (1, 2) {
    (0, n) -> // this case will never be reached
    (1, n) -> print("Case reached!");
    _ -> // default case that matches everything
  }
\end{verbatim}

\subsubsection{Delimitation}

Due to other work being deemed more importantly, match-statements will not be further developed. A future improvement to TLDR could possibly include match statements.
