\subsubsection{Functions}
\label{subsec:functions}

For the representation of functions in our language, we wanted to stay as close as possible to the mathematical functions, such as \enquote{f(x,y) = z}. In order to achieve this, the type of the function arguments had to either be inherent or declared elsewhere. Since we chose to 

\kanote{moar text needed plz}

Functions are declared syntactically as follows.

\begin{verbatim}
  let <funcName>(<parameterList>) : <typeSignature> := {<body>};
\end{verbatim}

A concrete example:

\begin{verbatim}
  let plus(x, y) : Int -> Int -> Int := {x + y};
\end{verbatim}

The syntax for invoking a function is the following.

\begin{verbatim}
  <funcName>( <parameter1>, <parameterN> )
\end{verbatim}

The semantics for declaring a function is as follows:

\begin{align*}
&\inference[$\text{INIT}_{FUNC}$]{}
                         {\Braket{\Tlet \Tx(y) := \{S\};,e} \Rightarrow_S e'}
												 {, e' = e[\Tx \mapsto \Braket{S,y}]}
\end{align*}