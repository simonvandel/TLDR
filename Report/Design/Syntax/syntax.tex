\section{Syntax and Semantics}
\label{sec:syntax}
In this section, will present and discuss some considerations made regarding the syntax and semantics of this language. The aim of this section is to provide insight into the constructs of the language, why they have taken their form.

The primitive data types and operators will be described in \cref{subsec:primitives,subsec:primitiveOps}. After this, functions and their declarations are explained in \cref{subsec:functions}. Next, comments will be briefly mentioned in \cref{subsec:comments}, followed by assignment, or binding, of values to symbols in \cref{subsec:value_assignment}. After this, more complex syntactical constructs, such as actors, structures, loops and selective control structures, will be explained in \cref{subsec:actors,subsec:structs,subsec:forLoopStatements,subsec:ifStatements}.

\section{Semantics}\label{semantics}
This section describes the semantics of the constructs in TLDR both formally and informally. It is split into pieces each describing one construct both formally and informally.

\subsection{Formal semantics in TLDR}
To describe the semantics formally it was decided to use small-step semantics. The alternative was big-step semantics but since the language depends heavily on non-determinism and parallelism and this is not possible to represent in big-step semantics, small-step semantic was a better choice.

The formal rules have three parts as illustrated in \cref{SS-semantics}

\begin{figure}[H]
\begin{align*}
&\inference[$RULE-NAME$]{[PREMISE]}
												{[CONCLUSSION]}
												{,[SIDE-CONDITION]}
\end{align*}
\caption{A desription of formal small-step semantics}
\label{SS-semantics}
\end{figure}

\subsection{Actor-Environment model}
The formal semantics will also use an actor-environment model.

$a = \text{Anames} \cup \{next\} \rightharpoonup e \times st$

$e = \text{Symbols} \rightharpoonup \text{Stm} \times \text{Symbols}$
$st = \text{Structs} \rightharpoonup \text{Stm}$
$at = \text{ActorTypes} \rightharpoonup \text{Stm}$

\subsection{General Constructs}
\begin{align*}
&\inference[$\text{INVOKE}$]{\Braket{S_1,e[p_1' \mapsto \Braket{S_2,p_2'}]} \Rightarrow_S \Braket{S_1',e'}}
                  {\Braket{p_1(p_2),e} \Rightarrow_S \Braket{S_1',e'}}
\\
&									{,e(p_1) = \Braket{S_1,p_1'}, e(p_2) = \Braket{S_2,p_2'}}
\end{align*}

For this language we have decided upon certain syntactical constructs. Here follows the choiced made and the reasoning behind them.

% brackets
% semi-kolon
% comments
% value of assignments
%

Logical comparisons.

For logical comparisons we chose \enquote{=}. This was done in accordance with the goal of keeping a natural mathematic language. In mathematics \enquote{=} is read as \enquote{is equal to} or simply \enquote{equals}, and is used for stating that two parts are equivalent to eachother. Sometimes mathematicians use this a statement in a contradicting manner, where they expect to prove the statement to be false. It is from this perspective of being a statement, either true or false, that we chose \enquote{=} to be a logical comparison.

Assignment

In traditional mathematical notation, the \enquote{=} symbol is also sometimes used to let certain symbols represent a more complex meaning, in order to simplify something, such as an equation or a function. When used like this, often mathematicians put the word \enquote{let} in front of a statement to denote that it is a definition. We have chosen to follow this construct as well, letting immutable assignment be denoted in this fashion, \enquote{let x = 2}, since they are comparable to definitions.

For assigning mutable variables we chose \enquote{<-}. This concept is less known in traditional mathematics, but is widely used in computational science. In the historically significant languages Fortran and C, the \enquote{=} symbol, was used for this. However, since we wish to keep that symbol closer to its original meaning, we needed something else. \enquote{<-} was chosen, since it is a known symbol from other languages, and also since the arrow symbolises transition or transformation. The asymmetry of the \enquote{<-} symbol also illustrates that it matters which side of the symbol a variable is on, as opposed to the \enquote{=} symbol.


Initialisation/Type decleration

Since mathematics is theortical abstraction, it is no problem to denote large numbers, precise numbers or even infinities. Computers, sadly, have pshyical limits which makes it complicated or ineffecient to express all these things as the same. This is the main reason for type declerations of numbers. In mathematics we can also denote a number to be of a certain \enquote{type}, such as the natural numbers, the rational numbers and the real numbers. However in mathematics, this is done for a different reason. However, both computer science's type declerations and mathematics' number systems, serve to provide a set of characteristics for the number, and how it will behave. We where inspired by this similarity, and chose to preserve it in our language. However, The mathematical symbol for \enquote{set membership} isn't found on a regular keyboard, so instead we chose the colon \enquote{:}.

When initialising a a variable, it being either muteable or immutable, the programmer must declare its type, ..

\subsection{Primitive Data Types}
\label{subsec:primitives}

Since mathematics is theoretical abstraction, it is rarely a problem to denote size of numbers, precision on numbers or even infinities. Computers, sadly, have psychical limits which makes it complicated or inefficient to express all these things as just \enquote{numbers}. This the main reason for type declarations on numbers. In mathematics we can also denote a number to be of a certain \enquote{type}, such as the natural numbers, the rational numbers and the real numbers. However in mathematics, this is done for a different reason, namely the different charateristics numbers have.

However, both computer science's type declarations and mathematics' number systems, serve to provide a set of characteristics for the number, and how it will behave. We were inspired by this similarity, and chose to preserve it in our language. However, The mathematical symbol for \enquote{set membership} isn't found on a regular keyboard, so instead we chose the colon \enquote{:}.

When initialising a variable, it being either mutable or immutable, the programmer must declare its type, as if the variable belonged to a set, where the set must be a type in our language.

A primitive is a predefined value type that cannot be constructed using other value types.

The following primitives exist in the language. All primitives are lower-cased.

\begin{itemize}
  \item int
  \item real
  \item bool
  \item char
  \item unit
\end{itemize}

\paragraph{Integer}
\label{subsubsec:int}

The int primitive can have values of 0 to \sinote{hvor mange bits?}. The int primitive's literal representation is as whole numbers e.g. \emph{2}.

\paragraph{Real}
\label{subsubsec:real}

The real primitive can have values of \sinote{hvordan definerer vi det her?}. The real primitive's literal representation is as decimal numbers with fractions e.g. \emph{2.5} or \emph{2.0}.

\paragraph{Bool}
\label{subsubsec:bool}

The bool primitive's value can be either \emph{true} or \emph{false}.

\paragraph{Char}
\label{sec:char}

The char primitive can have values of the ASCII standard. It is written literally as a character defined in the ASCII standard, surrounded by single quotation marks. For example: \emph{ '0' } and \emph{ 'A' }.

\paragraph{Unit}
\label{sec:unit}
\kanote{overvej om vi kan forklare unit type uden syntax}

The unit primitive can have only one value: itself. The use of the primitive is to signal emptiness.

From the start the project group aimed towards a language with type inference, due to types not being a known artifact/property in the mathematics of physics and social sciences, which is the scope of the project. Although due to prioritisation and time span of the project, it was decided that the language should initially have an explicit type system.

Types not being known an artifact in previously mentioned sciences mathematics, a technique known from the functional paradigm of separating the type signature from the function definition, as shown here \cref{typesignature}, was chosen. In this signature the function can take any number of arguments and return either a single typed symbol or a new function. The input and output is differenciated by the amount of input parameters (arg1, arg1, ... ,argN) where N would be the amount of inputs and M - N would be the size of the output.

\begin{verbatim}
  functionName(arg1, arg1, ... ,argN) :: arg1Type -> arg2Type -> ... -> argMType
\end{verbatim}
\label{typesignature}

If the user wants to use functions instead of single typed symbols he or she simply puts these in parentheses as seen in \cref{typesignatureexample0}. Here the function takes another function that maps an integer to an integer and maps this to a real.

\begin{verbatim}
  f(x) :: (int -> int) -> real
\end{verbatim}
\label{typesignatureexample0}

This can of cause be nested in as many levels as the user decides as illustrated in \cref{typesignatureexample1}, where the function takes a function typed the same way as in \cref{typesignatureexample0} and maps this to a list of characters. Note that both \cref{typesignatureexample0} and \cref{typesignatureexample1} still only takes one argument (x) sinco this is a function and can be used as such in the body.

\begin{verbatim}
  f(x) :: ((int -> int) -> real) -> [char]
\end{verbatim}
\label{typesignatureexample1}

This approach proposed a problem with the concept of \enquote{unit}. This represents the non-existing value, for the type system. It means that something does not have a value and using it in any context does not make sense. For instance when a function returns nothing ie unit, it cannot be assigned to anything. The need for explicitness resides in that The Language Described in This Report have functions as, what is called, \enquote{first class citisens}, meaning that functions can be used as arguments and essentially are treated in the same way as other symbols such as integers, structs ect. Here the language becomes umbigious with implicit unit, ie the user not needing to explicitly tell when a function takes or returns unit. For instance \cref{typesignatureexample0} we would not be able to tell wether the function takes a function that takes an int and returns an int or returns a function that goes from int to unit since the syntax is the same.
Several of proposed solution will be described here after.

First the type signature of this language looks as follows:
\begin{verbatim}
  printint(a): int -> int
\end{verbatim}
Where the first $int$ is the parameter $a$ of the function $printint$, and the second $int$ of the signature is the return type.

So in the case that the function $printint$ should not return anything, the last type of the type signature will be decorated with a type of nothing, e.g. unit, as follows:
\begin{verbatim}
  printint(a): int -> unit
\end{verbatim}

First proposal of a solution is to denote unit as \enquote{nothing}, like this:
\begin{verbatim}
  printint(a): int ->
\end{verbatim}
but this is ambiguent, in that white space, \enquote{nothing}, if the type signature of a function is that it takes nothing as parameter but returns something, like this:
\begin{verbatim}
  printint(): -> int
\end{verbatim}
.... %Expand on this

\kanote{var 'ugly' vores argument?}

So solve this ambiguity, one can introduce a new symbol to indicate return types $=>$, this was considered ugly and unnecessary by the project group though. So in the end the project group chose to explicitly express unit to avoid confusion for users of the language. The explicit format ended up as follows:
\begin{verbatim}
  printint(a): int -> unit
\end{verbatim}


\subsection{Value Assignment}
\label{subsec:value_assignment}

A value can be assigned to a symbol in either a constant binding or a variable binding.

In traditional mathematical notation, the \enquote{=} symbol is also sometimes used to let certain symbols represent a more complex meaning, in order to simplify something, such as an equation or a function. When used like this, often mathematicians put the word \enquote{let} in front of a statement to denote that it is a definition. We wanted to follow this construct as well letting immutable assignment be denoted in this fashion, since they are comparable to definitions. 

But since we wanted assignments, be it mutable or immutable, to have a similarities, we chose \enquote{:=} for all assignments. This concept is less known in traditional mathematics, but is widely used in computational science. In the historically significant languages Fortran and C, the \enquote{=} symbol, was used for this. However, since we wish to keep that symbol closer to its original meaning, we needed something else. \enquote{:=} was chosen, since it is a known symbol from other languages. The asymmetry of the \enquote{:=} symbol also illustrates that it matters which side of the symbol a variable is on, as opposed to the \enquote{=} symbol.

The grammar for value assignments ended up following this grammar:

\kanote{here goes grandmar}

as can be seen, values can either be an immutable constant, or a muteable variable.

A constant binding can never have its value changed. For example, if \enquote{a} is bound to \enquote{5}, \enquote{a} can never refer to another value than \enquote{5} in the same lexical scope. 

The syntax for constant value assignment is as follows.
\begin{verbatim}
  let <symbolName> : <type> := <value>
\end{verbatim}
A concrete example:

\begin{verbatim}
  let x : int := 2
\end{verbatim}

A variable binding can always change the value it refers to. For example, if \enquote{b} is bound to \enquote{2}, it is perfectly possible to later in the source code refer to {10}.

The syntax for variable value assignment is as follows.

\begin{verbatim}
  var <symbolName> : <type> := <value>

// a later reassignment
  <symbolName> := <value>
\end{verbatim}
A concrete example:

\begin{verbatim}
  var a : int := 2
  
// a later reassignment
  a := 2
\end{verbatim}

The semantics for declaring, initializing and assigning variables is as follows:
\kanote{semantik: decls er udkommmenteret. Hvorfor? er det ikke korrekt?}

\begin{align*}
%&\inference[$\text{DECL}$]{}
%                         {\Braket{\Tlet x := a,e,st} \Rightarrow_S \Braket{x := a, e',st}}
%												{, l = e(next), e' = e[x\mapsto l, next\mapsto new(l)]}
%\\\\
&\inference[$\text{INIT}_{SYM}$]{}
                         {\Braket{\Tlet \Tx := \{S\};,e} \Rightarrow_S e'}
												 {, e' = e[\Tx \mapsto \Braket{S,\epsilon}]}
\\\\
&\inference[$\text{INIT}$]{}
                         {\Braket{\Tvar \Tx := S,e} \Rightarrow_S e'}
												 {, e' = e[\Tx \mapsto \Braket{S,\epsilon}]}
\\\\
&\inference[ASS]{}
                 {\Braket{\Tx := S,e} \Rightarrow_S e'}
								 {, e' = e[\Tx \mapsto \Braket{S,\epsilon}]}
\end{align*}

\subsection{Primitive Operators}
\label{sec:primitiveOps}

Primitive operations are the operations that are build into the language.

\subsubsection{Mathematical Operators}
\label{sec:mathOps}

The following mathematical operations are built into the language.

\begin{itemize}
  \item \textbf{+} A binary operator that adds two numbers of the same type
  \item \textbf{-} A binary operator that subtracts two numbers of the same type
  \item \textbf{/} A binary operator that divides two numbers of the same type
  \item \textbf{\%} A binary operator that integer divides two number of the same type and returns the remainder
  \item \textbf{\^}  A binary operator that lifts the first number to the power of second number
  \item \textbf{root} A binary operator that roots the first value to second value
\end{itemize}

\sinote{Vi skal nok forklare hvilke nummer typer der kan plusses sammen osv.}

\subsubsection{Logical Operations}
\label{sec:logicOps}

\begin{itemize}
  \item \textbf{=} A binary operator that compares the two values for equality. Returns true if equal. False otherwise.
  \item \textbf{!=} A binary operator that compares the two values for equality. Returns true if not equal. False otherwise.
  \item \textbf{<} A binary operator that compares the two values. Returns true if first value is strictly less than second value. False otherwise.
  \item \textbf{<=} A binary operator that compares the two values. Returns true if first value is less than or equal to second value. False otherwise.
  \item \textbf{>} A binary operator that compares the two values. Returns true if first value is strictly greater than the second value. False otherwise.
  \item \textbf{>=} A binary operator that compares the two values. Returns true if first value is greater than or equal to the second value. False otherwise.
\end{itemize}

\subsubsection{Boolean Operations}
\label{sec:boolOps}

The following operations only operate on bool types.

\begin{itemize}
  \item \textbf{AND} A binary operator that returns true if both values on both sides are true. False otherwise
  \item \textbf{OR} A binary operator that returns true if one value on either sides are true. False otherwise
  \item \textbf{XOR} Same as \textbf{OR}, but in the case that both sides are true, returns also false.
  \item \textbf{NOT} Negates the boolean value
  \item \textbf{NAND} Same as NOT (<bool1> AND <bool2>)
\end{itemize}


\subsection{Functions}
\label{subsec:functions}

Functions can be declared in two ways. By separating the type signature and the function body or by combining the signature and the function body.

\paragraph{Separated function declaration}

The syntax for the separated function declaration is as follows. The type signature and the body must be declared in the same lexical scope.

\begin{verbatim}
  <funcName> : <typeSignature>;
  let <funcName>(<parameterList>) := {<body>};
\end{verbatim}

A concrete example:

\begin{verbatim}
  plus : int -> int -> Int;
  let plus(x, y) := {x + y};
\end{verbatim}


\paragraph{Combined function declaration}

The syntax for the combined function declaration is as follows.

\begin{verbatim}
  let <funcName>(<parameterList>) : <typeSignature> := {<body>};
\end{verbatim}

A concrete example:

\begin{verbatim}
  let plus(x, y) : Int -> Int -> Int := {x + y};
\end{verbatim}

\subsubsection{Comments}
\label{subsec:comments}

Comments are declared in C style by \enquote{//} being a single line comment and \enquote{/**/} being a multi line comment.

A concrete example:

\begin{verbatim}
  // this is a single line comment
  /* this is
     a
     multi line comment */
\end{verbatim}


\section{Actors}

Here follows descriptions of the usage of actors in TLDR. This includes different principles, functionalities, syntactical and the semantics. But before we can discuss the use of actors in TLDR, we must first cover the semantics of the parallelism used.

\subsubsection{Semantics of parallelism}

\kanote{forklaring af parallisms semantic}

\begin{align*}
%%%%%%%%%%%%%%%%%%%%%%%%%%%%%%%%%%%%%%%%%%%%%%%%%%%%%%%%%%%%%%%
\intertext{The left side of a parallel statement is executed, but not finished. The environment and actor model is updated when executing $S_1$.}
&\inference[$\text{PAR}_1$]{\Braket{S_1,e_1,\Ta} \Rightarrow_S \Braket{S_1',e_1',\Ta'}} 
                           {\Braket{S_1,e_1,\Ta}|\Braket{S_2,e_2,\Ta} \Rightarrow_S \Braket{S_1',e_1',\Ta'}|\Braket{S_2,e_2,\Ta'}}
%%%%%%%%%%%%%%%%%%%%%%%%%%%%%%%%%%%%%%%%%%%%%%%%%%%%%%%%%%%%%%%
\intertext{The left side of a parallel statement is executed, and finishes. The environment and actor model is updated when executing $S_1$.}
&\inference[$\text{PAR}_2$]{\Braket{S_1,e_1,\Ta} \Rightarrow_S \Braket{e_1',\Ta'}} 
                           {\Braket{S_1,e_1,\Ta}|\Braket{S_2,e_2,\Ta} \Rightarrow_S \Braket{S_2,e_2,\Ta'}}
%%%%%%%%%%%%%%%%%%%%%%%%%%%%%%%%%%%%%%%%%%%%%%%%%%%%%%%%%%%%%%%
\intertext{The right side of a parallel statement is executed, but not finished. The environment and actor model is updated when executing $S_2$.}
&\inference[$\text{PAR}_3$]{\Braket{S_2,e_1,\Ta} \Rightarrow_S \Braket{S_2',e_1',\Ta'}} 
                           {\Braket{S_1,e_1,\Ta}|\Braket{S_2,e_2,\Ta} \Rightarrow_S \Braket{S_1',e_1',\Ta'}|\Braket{S_2,e_2,\Ta'}}
%%%%%%%%%%%%%%%%%%%%%%%%%%%%%%%%%%%%%%%%%%%%%%%%%%%%%%%%%%%%%%%
\intertext{The right side of a parallel statement is executed, and finishes. The environment and actor model is updated when executing $S_2$.}
&\inference[$\text{PAR}_4$]{\Braket{S_2,e_1} \Rightarrow_S \Braket{e_1',\Ta'}}
                           {\Braket{S_1,e_1,\Ta}|\Braket{S_2,e_2,\Ta} \Rightarrow_S \Braket{S_1',e_1',\Ta'}}
%%%%%%%%%%%%%%%%%%%%%%%%%%%%%%%%%%%%%%%%%%%%%%%%%%%%%%%%%%%%%%%
\end{align*}

\subsubsection{Isolation and Independence}

A central principle in the programming language described in this report is the use of actors, based on actor-modelling. Actors are to be seen as entities with interaction. In other words, in order for a construct to qualify as an actor, it must define a way to behave when other actors interact with it. Actors should function independently, and in that regard, not be open to direct manipulation and only able to be changed through the messages it receives. This requirement is due to the wish of separation of processes, which will allow for greater concurrency by letting processes operate on local data instead of global, shared data. Therefore TLDR tries to encourage natural isolation of functionality through actors, which in turn will also give a greater control of race conditions as no data is ever accessed by more than one process.

\subsubsection{Main is an Actor}

Main is always the first actor, and any program writtin in TLDR is started with main receiving the arguments message. This is done to force the programmer to start and end the program with an actor, which will better support the actor modelling perspective. This means that the main actors is started with a message for the arguments, and when the main actor is killed the program will stop executing, whether or not there are still working actors.

Another way that this affects the programmer, is the idea of spawning actors and having them send messages to main, instead of calling functions and having them return. This also means that there must be a way to reference main, since it is not spawned by another actor. This is solved by the introduction of the \enquote{me} keyword, which will evaluate to a reference to the current actor. This is very useful, especially if an actor wishes to delegate work to other actors. The message that is sent with the work, must simply contain a reference back to the delegating actor, which was included through the use of the \enquote{me} keyword.

Here are the semantics for the main actor.

\begin{align*}
&\inference[$\text{MAIN}$]{input \mapsto \Braket{S,\epsilon}}
                          {\Braket{\Treceive \Tr:args := \{S\};,e} \Rightarrow_S \Braket{\Tr := input;S, e]}}
\end{align*}

\subsubsection{Construction of an Actor}
\label{sub:constructionOfAnActor}

The syntactical declaration of an actor is as follows:

\begin{lstlisting}
actor <identifier> := {
  <functionality>
}
\end{lstlisting}

And a concrete example could be:

\begin{verbatim}
  actor earth := {
    var temperature:real := 0;
    receive sunlight:light := {
      temperature := temperature + 0.1;
    }
  }
\end{verbatim}

As shown, the actor keyword precedes the definition, denoting the meaning of said definition. After the keyword an identifier of the declaration is needed, which will serve as the of type the actor. It is suggested that this identifier reflects the role of the actor in a context of use. Noticeably there is no \enquote{let} or \enquote{var} keyword in front of the definition, as there usually is when assigning. This is a deliberate choice since \enquote{let} and \enquote{var} implies interchangeability, which is not an option is this case. If variable declaration of actor definitions were possible, it would effectively be the equivalent of changing the definition of a type on run-time, which would make little sense, and completely undermine the type safety.

Some general semantics of actors:

\begin{align*}
&\inference[TYPEOF]{}
                  {e \vdash m \Rightarrow_T t}
                  {, \mathbb{T}(m) = t}
\\\\
&\inference[$\text{ACTOR}$]{a' = a[\Tact \mapsto e \times st]}
                           {\Braket{\Tactor \Tact := \{S\}, a} \Rightarrow_S \Braket{S,e,at[\Tact \mapsto S],a'}}
\\\\
&\inference[$\text{ACTOR}$]{}
                           {\Braket{\Tactor \Tact := 1S, at} \Rightarrow_S \Braket{at[\Tact \mapsto S]}}
\end{align*}

\subsubsection{Basic actor functionality}

There are four basic functionalities: \enquote{spawn}, \enquote{die}, \enquote{send}, and \enquote{receive}, which are all used through keywords. It was desired to keep the syntax of these functionalities different from the syntax of functions. Even though they behave much like functions, taking input and giving output, they are more powerful. Regular functions cannot contain a type as a parameter, but the \enquote{spawn} functionality does this. Due to these and more differences, which follows below, we decided to separate them syntactically.

The \enquote{spawn} functionality is used to create new instances of actors. And example could be:

\label{actorfuncSpawn}
\begin{lstlisting}
actor <identifier> := {
 <functionality>
}

let MyActor:<identifier> := spawn <identifier> <message>;

or alternatively:

var MyActor:<identifier> := spawn <identifier> <message>;
\end{lstlisting}

As can be seen above, there are four parts of a spawn functionality: an identifier, the keyword, type of actor, and possibly a message. Firstly the identifier is preceded by a keyword for mutability, such as \enquote{let} or \enquote{var}. This allows for the substitution of handles, which provides possibilities of dynamic changes. This however opens up the possibility of \enquote{losing contact} with an actor, if the handle is replaced. This could potentially lead to memory leaks if not handled properly. \kanote{reference til garbage collection afsnit}

When spawning a new actor, you can also choose to add a message. The reason for this is to give the programmer a way of initialising the new actor with a certain message. In object-oriented languages this is usually done with a constructor, however including this would conflict with a central principle, since it would be a manipulation of an actor without a message.

The semantics of \enquote{spawn} is as follows:

\begin{align*}
&\inference[$\text{SPAWN}$]{}
                       {\Braket{\Tlet \Tact:T := \Tspawn \; T \Tm,\Ta} \Rightarrow_S \Braket{\Tsend \Tact \Tm,\Ta[act \mapsto e \times st]}}
\\
&                       {, \Ta(x) \mapsto \Braket{S,p} , \Ta' = \Ta[\Tact \mapsto \Braket{S, p}]}
\end{align*}

After an actor has been spawned, it will be possible to send messages to it. This is done with the \enquote{send} keyword. This could be done as follows:

\label{actorfuncSend}
\begin{lstlisting}
MyMsg:int := 42;

Send MyActor MyMsg;
\end{lstlisting}

It is also possible for actors to send messages to themselves. Such messages will be treated as any other message.

The semantics of \enquote{send} is as follows:

\begin{align*}
&\inference[$\text{SEND}$]{e_2 = a(act),m \Rightarrow_T t}
                       {\Braket{\Tsend \Tact \Tm ; S,\Ta,e_1} \Rightarrow_S \Braket{S,e_1,a}|\Braket{\_t(m),e_2,a}}
\end{align*}

When an actor is sent a message, it must act according to a defined a way of handling that type of message. This definition is declared with the \enquote{receive} keyword, and can be seen below:

\label{actorfuncReceive}
\begin{lstlisting}
actor <nameOfActor> := {
 receive <nameOfMessage>:<typeOfMessage> := {
  <functionality>
 }
}
\end{lstlisting}

In this example the receive defines the way messages of the type \enquote{<typeOfMessage>} are handled. Within the functionality \enquote{<nameOfMessage>} is the reference to the message, and this message is immutable no matter if it was mutable where it was sent from. This is done to discourage further use of old messages.

It is also the intention to include a \enquote{wait on <typeOfMessage>} keyword, which will cause the actor to not evaluate the next message in the messagequeue, but instead traverse the queue, until a message matching \enquote{<typeOfMessage} is found. Then that message is de-queued and evaluated. This has not been implemented in the current release of TLDR, but it will be a central part of supporting discrete simulations.

The semantics of \enquote{receive} is as follows:

\begin{align*}
&\inference[$\text{RECEIVE}$]{}
                           {\Braket{\Treceive r:t := \{S\};,e} \Rightarrow_S \Braket{e[\_t \mapsto \Braket{S,r}]}}
\end{align*}

When an actor is no longer needed, it is possible to discard it with the \enquote{die} keyword. As opposed to other languages where \enquote{die} is called by a parent, that is the actor that spawned the actor, \enquote{die} can only be called by the actor itself. This is done in order to keep the principle of only having actors react to messages. When an actor dies it stops immediately and does not compute further, and what ever messages might have been in the queue, will be lost.
%mainfile: ../../master.tex
\subsection{Structures}
\label{subsec:structs}

In TLDR there are three ways to do encapsulation. Actors, Tuples and structs. Structs are unique by being fully accessible within the scope, and having named fields. Structs are especially useful in TLDR for creating messages.

\subsubsection{Defining Structures}
\label{sec:defStructures}

\subsubsection{Syntax}

Structures are defined by using the \enquote{struct} keyword. The grammar for declaring structs are as follows:

\begin{grammar}
  <Struct> ::= 'struct' <Identifier> ':= \{' <TypeDecls> '\}'
\end{grammar}

And a concrete example:

\begin{lstlisting}[style=TLDR]
  struct Person := {Name:[char]; Age:int}
\end{lstlisting}

\subsubsection{Semantics}

having these semantics:

\begin{align*}
\intertext{In the case that we have multiple S of assignments, we can rewrite the T to now map to new s' that includes x, in the new st' environment and the rest of declaration statements}
&\inference[$\text{STRUCT}$]{}
                            {\Braket{\Tstruct T := \{x:T';S\}, env_s, st} \Rightarrow_S \Braket{\Tstruct T := \{S\},env_s,st'}}
\\
&{\Twhere st' = st[T \mapsto s'],st(T) = s,s' = s \cup x]}
\\\\
\intertext{In the case that we have multiple S of assignments, we can rewrite the T to now map to new s' that includes x, in the new st' environment}
&\inference[$\text{STRUCT}$]{}
                            {\Braket{\Tstruct T := \{x:T\}, env_s, st} \Rightarrow_S \Braket{env_s,st'}}
\\
&{\Twhere st' = st[T \mapsto s'],st(T) = s,s' = s \cup x]}
\end{align*}

\begin{align*}
&\inference[$\text{STRUCT}$]{}
                            {\Braket{\Tstruct T_1 := \{x:T_2;S\}, env_s} \Rightarrow_S \Braket{\Tstruct T_1 := \{S\},env_s'}}
\\
&{\Twhere env_s' = env_s[T \mapsto \Braket{\epsilon,s'}],env_s(T) = s,s' = s \cup x]}
\\\\
&\inference[$\text{STRUCT}$]{}
                            {\Braket{\Tlet \; x:T := (S),env_s} \Rightarrow_S \Braket{x:T := (S),env_s'}}
\\
&{env_s' = env_s[x \mapsto (S,s)],env_s(T) = (\epsilon,s)]}
\\\\
&\inference[$\text{STRUCT}$]{}
                            {\Braket{s:T := (f := x;S),env_s} \Rightarrow_S \Braket{s.f := x;s:T := (S),env_s}}
\end{align*}

\subsubsection{Type Rules}

\begin{align*}
\intertext{The elements can be of any type defined in $\Tt$}
&\inference[STRUCT]{E[s \mapsto (e_1:\Tt_1;e_2:\Tt_2;...;e_n:\Tt_n) \rightarrow ok]\vdash S : ok & }
                 {\Tenv \mathbin{\text{struct s}} := \{e_1:\Tt_1;e_2:\Tt_2;...;e_n:\Tt_n\}; S: ok}
\end{align*}



\subsubsection{Initialising Structures}
\label{sec:initStructures}

Structs can be initialised and assigned to symbols using either a constant assignment or a variable assignment. Structs initialised as a constant assignment cannot change any of the fields of the structs; the struct is immutable. Structs initialised as a variable assignment can change all of its fields at any time; the struct is mutable.

\subsubsection{Syntax}

The syntax for initialising a struct is as follow.

\begin{grammar}
<StructLiteral> ::= '(' (<Reassignment>';')* ')' (':' <Identifier>)?
\end{grammar}


With concrete examples for immutables:

\begin{verbatim}
  let Alice:Person := (Name := "Alice"; Age := 20);
\end{verbatim}

And for mutable struct is as follow.

\begin{verbatim}
  var Alice:Person := (Name := "Alice"; Age := 20);
\end{verbatim}

And for usage in lists.

\begin{verbatim}
  [(Name := "Alice"; Age := 20):Person];
\end{verbatim}

\subsubsection{Access to Structure Fields}
\label{sec:accessStructFields}

Fields can be access using the following syntax.

\begin{verbatim}
  Alice.Name; // "Alice"
\end{verbatim}

Structs declared as mutable can have fields reassigned using the following syntax.

\begin{verbatim}
  Alice.Name; // "Alice"
  Alice.Name := "Bob";
  Alice.Name; // "Bob";
\end{verbatim}
 
\subsubsection{Semantics}

\begin{align*}
\intertext{In the case that we have multiple S of assignments, we can rewrite the s to now having an accessor that maps to value of the first assignment and the rest of pending assignments in s}
&\inference[$\text{STRUCT}$]{}
                            {\Braket{\Tlet \; s:T := (f := x;S),env_s,st} \Rightarrow_S \Braket{s.f := x;\Tlet \; s:T := (S),env_s,st}}
\\\\
\intertext{In the case that we have only one assignment, we can rewrite the s to now having an accessor that maps to value of the assignment}
&\inference[$\text{STRUCT}$]{}
                            {\Braket{\Tlet \; s:T := (f := x),env_s,st} \Rightarrow_S \Braket{s.f := x,env_s,st}}
%\\\\
%&\inference[$\text{STRUCT}$]{}
%                            {\Braket{(f := x;S),env_s,st} \Rightarrow_S \Braket{s.f := x;\Tlet \; s:T := (S),env_s,st}}
%\\\\
%&\inference[$\text{STRUCT}$]{}
%                            {\Braket{(f := x):T,env_s,st} \Rightarrow_S \Braket{s.f := x,env_s,st}}
\end{align*}

\subsubsection{Type Rules}

Each e of type t is matching the declared struct types, evaluates to the type of the declared struct.

\begin{align*}
&\inference[STRUCTLITERAL]{\Tenv (e_1 : \Tt_1;e_2 : \Tt_2;...;e_n : \Tt_n):\Tt'}
                 {\Tenv \mathbin{\text{(}} e_1; e_2;...;e_n\mathbin{\text{)}}:\Tt': \Tt'}
\end{align*}

\subsubsection{For-loop Statements}
\label{subsec:forLoopStatements}

There are two for-loop statements defined in the language. A for-in loop and a variation of the for-in loop where both the index and the element of the collection being looped is provided.

\paragraph{For-in loop}
\label{sec:forInLoop}

The for-in loop statement has the following syntax.

\begin{verbatim}
  for <element> in <collection> {<loopBody>}
\end{verbatim}

A concrete example:

\begin{verbatim}
  for i:int in [0..10]:[int] { /* Do stuff with i elements */ }
\end{verbatim}

\paragraph{For-in loop with index}
\label{sec:forInLoopIndex}

A variation of the for-in loop that also provides the index of the current element, has the following syntax.

\begin{verbatim}
  for (<index>, <element>) in <collection> {<loopBody>}
\end{verbatim}

A concrete example:

\begin{verbatim}
  for (index:int, elem:int) in [0..10]:[int] { 
    /* Do stuff with i elements and index */ 
  }
\end{verbatim}

%mainfile: ../master.tex
\subsection{While-loop}
\label{subsec:whileLoopStatements}

The while loop statement runs a block of statements until a boolean expression provided returns false.

\subsubsection{Syntax}

\begin{grammar}
<While> ::= 'while' <Expression> <Block>
\end{grammar}

A concrete example:

\begin{verbatim}
  var i:int := 0;
  while (i < 10) {i := i + 1}
\end{verbatim}

\subsubsection{Semantics}

\begin{align*}
\intertext{In the case that the boolean expression $b$ is true, the rule for composition of statements is used. First the statement $S$ is evaluated and then the while statement is run again. The statement $S$ may have changed $b$.}
&\inference[$\text{WHILE}_\top$]{env_S \vdash b \Rightarrow_B \top}
                       {\Braket{\Twhile(b)\{S\},env_s} \Rightarrow_S \Braket{\{S\}; \Twhile (b)\{S\},env_S}}
\\\\
\intertext{In the case that the boolean expression $b$ is false, the while statement has ended, and can simply be rewritten to the environment $env_S$}
&\inference[$\text{WHILE}_\bot$]{env_S \vdash b \Rightarrow_B \bot}
                       {\Braket{\Twhile(b)\{S\},env_S} \Rightarrow_S env_S}
\end{align*}

\subsubsection{Type Rules}

\begin{align*}
\intertext{The conditional body of the while construct must be of type bool. The body of the while-loop can be of any type defined in $\Tt$}
&\inference[WHILE]{\Tenv b : \Tbool &
                  \Tenv e : \Tt}
                 {\Tenv \mathbin{\text{while}} \; (b) \; {e}: ok}
\end{align*}

\subsection{If-statements}
\label{subsec:ifStatements}

If-statements can be written either as a if-then-else statement or just as an if-statement. The following shows either way.

\begin{verbatim}
  // if-then-else statement
  if (<condition>) {<what to do if condition is true>}
  else {<what to do if condition is false>}

  // if-statement
  if (<condition>) {<what to do if condition is true>}
\end{verbatim}

A concrete example:

\begin{verbatim}
  // if-then-else statement
  if (2 + 2 = 4) {"math works!"}
  else {"something is wrong here!"}

  // if-statement
  if (remainingTime < 10) {initiateCountdown()}
\end{verbatim}

\subsubsection{Match-statements}
\label{subsec:matchStatements}

Match statements can be seen as syntactical sugar for multiple chained if-statements. The syntax is as follows.

\begin{verbatim}
  match <whatToMatchOn> {
    <case1> -> <actionOnCase1>
    <case2> -> <actionOnCase2>
    ...
    <caseN> -> <actionOnCaseN>
  }
\end{verbatim}

A concrete example:

\begin{verbatim}
  match (1, 2) {
    (0, n) -> // this case will never be reached
    (1, n) -> print("Case reached!");
    _ -> // default case that matches everything
  }
\end{verbatim}

\paragraph{Delimitation}

Due to other work being deemed more importantly, match statements will not be further developed. A future improvement to TLDR could possibly include match statements.
\begin{grammar}
<statement> ::= <ident> ‘=’ <expr>
\alt ‘for’ <ident> ‘=’ <expr> ‘to’ <expr> ‘do’ <statement>
\alt ‘{’ <stat-list> ‘}’
\alt <empty>
<stat-list> ::= <statement> ‘;’ <stat-list> | <statement>
\end{grammar}
