\subsubsection{If-statements}
\label{subsec:ifStatements}
\newcommand{\Tif}{\text{if}}
\newcommand{\Telse}{\text{else}}

If-statements can be written either as a if-then-else statement or just as an if-statement. The following shows either way.

\begin{verbatim}
  // if-then-else statement
  if (<condition>) {<what to do if condition is true>}
  else {<what to do if condition is false>}

  // if-statement
  if (<condition>) {<what to do if condition is true>}
\end{verbatim}

A concrete example:

\begin{verbatim}
  // if-then-else statement
  if (2 + 2 = 4) {"math works!"}
  else {"something is wrong here!"}

  // if-statement
  if (remainingTime < 10) {initiateCountdown()}
\end{verbatim}

Below follows the semantics for the If-statements:

\begin{align*}
&\inference[$\text{IF}_\top$]{}
                      {\Braket{if(b)\{S\},e} \Rightarrow_S \Braket{\{S\},e}}
                      {,b \Rightarrow_B \top}
\\\\
&\inference[$\text{IF}_\bot$]{}
                      {\Braket{if(b)\{S\},e} \Rightarrow_S e}
                      {,b \Rightarrow_B \bot}
\\\\
&\inference[$\text{IF-ELSE}_\top$]{}
                      {\Braket{if(b)\{S_1\}else\{S_2\},e} \Rightarrow_S \Braket{\{S_1\},e}}
                      {,b \Rightarrow_B \top}
\\\\
&\inference[$\text{IF-ELSE}_\bot$]{}
                      {\Braket{if(b)\{S_1\}else\{S_2\},e} \Rightarrow_S \Braket{\{S_2\},e}}
                      {,b \Rightarrow_B \bot}
\end{align*}