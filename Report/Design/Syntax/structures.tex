\subsubsection{Structures}
\label{subsec:structs}

\paragraph{Defining Structures}
\label{sec:defStructures}

Structures are defined by using the \enquote{struct} keyword. 

\begin{verbatim}
  struct <structName> := {<structFields>}
\end{verbatim}

\begin{align*}
&\inference[$\text{STRUCT}$]{\Braket{S, e} \Rightarrow_S \Braket{S', e'}}
                            {\Braket{struct\;s := \{S\}, st} \Rightarrow_S \Braket{struct\;s := \{S'\}, st[s \mapsto e']}}
                            {e=st(s)}
\\\\
&\inference[$\text{STRUCT}$]{\Braket{S, e} \Rightarrow_S e'}
                            {\Braket{struct\;s := \{S\}, st} \Rightarrow_S \Braket{st[s \mapsto e']}}
                            {e=st(s)}
\\\\
&\inference[$\text{STRUCT}$]{}
                            {\Braket{x:T, e} \Rightarrow_S e[x]}
\end{align*}

A concrete example:

\begin{verbatim}
  struct Person := {Name:[char]; Age:int}
\end{verbatim}

\paragraph{Initialising Structures}
\label{sec:initStructures}

Structs can be initialised and assigned to symbols using either a constant assignment or a variable assignment. Structs initialised as a constant assignment cannot change any of the fields of the structs; the struct is immutable. Structs initialised as a variable assignment can change all of its fields at any time; the struct is mutable.

The syntax for initialising an immutable struct is as follow.

\begin{verbatim}
  let Alice:Person := (Name := "Alice"; Age := 20);
\end{verbatim}

The syntax for initialising a mutable struct is as follow.

\begin{verbatim}
  var Alice:Person := (Name := "Alice"; Age := 20);
\end{verbatim}

\paragraph{Access to Structure Fields}
\label{sec:accessStructFields}

Fields can be access using the following syntax.

\begin{verbatim}
  Alice.Name; // "Alice"
\end{verbatim}

Structs declared as mutable can have fields reassigned using the following syntax.

\begin{verbatim}
  Alice.Name; // "Alice"
  Alice.Name := "Bob";
  Alice.Name; // "Bob";
\end{verbatim}
