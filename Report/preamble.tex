\documentclass[a4paper,12pt,english,twoside]{report}

%%%% PACKAGES %%%%

\usepackage[T1]{fontenc}
\usepackage[utf8]{inputenc} % encoding=utf8
\usepackage{fourier} % font
\usepackage{amssymb}
\usepackage{amsmath}
\usepackage{pifont}
\usepackage[english]{babel} % language support - hyphenations etc.
\usepackage{graphicx} % insertion of images
\usepackage{listings} % source code
\usepackage{color}
%itemize spacing
\usepackage{enumitem}
\setlist[1]{itemsep=-5pt}
\usepackage[noend]{algpseudocode}
\usepackage{algorithmicx}
\usepackage{algorithm}
\usepackage{tikz}
\usepackage{tikz-qtree}
\usepackage[inference]{semantic}
\usepackage{braket}
\usepackage{array}
\usepackage[nounderscore]{syntax}

%%%%%%%flotte listings
\lstdefinelanguage{scala}{
  morekeywords={abstract,case,catch,class,def,%
    do,else,extends,false,final,finally,%
    for,if,implicit,import,match,mixin,%
    new,null,object,override,package,%
    private,protected,requires,return,sealed,%
    super,this,throw,trait,true,try,%
    type,val,var,while,with,yield},
  otherkeywords={=>,<-,<\%,<:,>:,\#,@},
  sensitive=true,
  morecomment=[l]{//},
  morecomment=[n]{/*}{*/},
  morestring=[b]",
  morestring=[b]',
  morestring=[b]"""
}

\definecolor{dkgreen}{rgb}{0,0.6,0}
\definecolor{gray}{rgb}{0.5,0.5,0.5}
\definecolor{mauve}{rgb}{0.58,0,0.82}

\lstdefinestyle{scala}{
  language=scala,
  aboveskip=2mm,
  belowskip=2mm,
  showstringspaces=false,
  columns=flexible,
  basicstyle={\small\ttfamily},
  numbers=none,
  numberstyle=\tiny\color{gray},
  keywordstyle=\color{blue},
  commentstyle=\color{dkgreen},
  stringstyle=\color{mauve},
  breaklines=true,
  breakatwhitespace=true,
  tabsize=2,
}

\lstdefinestyle{erlang}{
  language=erlang,
  aboveskip=2mm,
  belowskip=2mm,
  showstringspaces=false,
  columns=flexible,
  basicstyle={\small\ttfamily},
  numbers=none,
  numberstyle=\tiny\color{gray},
  keywordstyle=\color{blue},
  commentstyle=\color{dkgreen},
  stringstyle=\color{mauve},
  breaklines=true,
  breakatwhitespace=true,
  tabsize=2,
}

\lstset{language=[Sharp]C}
%Use the typeface Source Code Pro for listings
\usepackage[nottdefault]{sourcecodepro}
\lstset{basicstyle=\fontfamily{SourceCodePro-TLF}\footnotesize,breaklines=true,commentstyle=}
\usepackage{color}
\usepackage{booktabs}
\usepackage{xcolor}
\usepackage{float}
\usepackage[font=it,labelfont=bf]{caption}
\DeclareCaptionFont{white}{\color{white}}
\DeclareCaptionFormat{listing}{\colorbox{gray}{\parbox{\textwidth}{#1#2#3}}}
\captionsetup[lstlisting]{format=listing,labelfont=white,textfont=white}

\usepackage{subfig} % support for sidebyside figures

\usepackage[mode=multiuser,draft]{fixme} % fixme config
\FXRegisterAuthor{si}{asi}{Simon}
\FXRegisterAuthor{ka}{aka}{Kasper}
\FXRegisterAuthor{ta}{ata}{Terndrup}
\FXRegisterAuthor{al}{aal}{Alexander}
\FXRegisterAuthor{ch}{ach}{Christian}
\FXRegisterAuthor{je}{aje}{Jens}

\usepackage[ % citation support
bibencoding=utf8,
  hyperref=true,
  backend=biber,
  style=ieee,
  sorting=none
  ]{biblatex}
  \addbibresource{sources.bib}

\usepackage{csquotes} % recommended by biblatex
\usepackage{lastpage}

\sloppy

% http://tex.stackexchange.com/questions/42063/illogical-twoside-margins
% layout page auto-centering horizontally and allow for press binding
\usepackage[hcentering,bindingoffset=8mm]{geometry}

\usepackage{hyperref} % hyperlinks in pdf
\usepackage{cleveref} % clever references. Use with \cref...
\let\origref\cref
\def\cref#1{\emph{\origref{#1}}}
\let\Origref\Cref
\def\Cref#1{\emph{\Origref{#1}}}

  % ¤¤ Kapiteludssende ¤¤ %
\usepackage{titlesec}
\titleformat{\chapter}
  {\normalfont\LARGE\bfseries}{\thechapter}{1em}{}
\titlespacing*{\chapter}{0pt}{-2ex}{0ex}

\let\Oldpart\part
\newcommand{\parttitle}{}
\renewcommand{\part}[1]{\Oldpart{#1}\renewcommand{\parttitle}{#1}}

\usepackage{fancyhdr}
\usepackage{nopageno}
\setlength{\headheight}{52pt}

\fancypagestyle{fancyplain}{} % no ordinary headers and footers in pages with chapters
\fancyhead{} % clear default header
\fancyfoot{} % clear default footer

\renewcommand{\chaptermark}[1]{\markboth{\thechapter.\space#1}{}} % do not print "chapter"

\fancypagestyle{BasicStyle}{
	\fancyhead[OL, ER]{
		\parbox{17mm}{{\includegraphics[scale=0.25]{aau_logo_cropped.png}}}\parbox{45mm}{\textsc{Aalborg University}\\\textit{sw404f15}}}
	\fancyhead[OR, EL]{\parbox{56mm}{\nouppercase{{\textsc{\parttitle}} \\ \leftmark}}}

	\fancyfoot[OR, EL]{Page {\thepage} of \pageref{LastPage}}
	\fancyfoot[OL, ER]{May 27\textsuperscript{th}, 2015}
}

\pagenumbering{roman}
\pagestyle{empty} % Use empty at first
\setlength{\belowcaptionskip}{-12pt}
\raggedbottom
\hyphenation{fa-brik-ken}

%%%%%%%%%% macros
\newcommand{\actortable}[2] {
%\vspace{1cm}
\begin{center}
\begin{minipage}{0.9\textwidth}
    \rule{\textwidth}{1pt}
    \begin{center}
      \vspace{-2ex}
        \textbf{#1}
      \vspace{-2ex}
    \end{center}
    \rule{\textwidth}{1pt}
    #2

    \rule{\textwidth}{1pt}
\end{minipage}
\end{center}
%\vspace{1cm}
}

\newcommand*{\blankpage}{
\vspace*{\fill}
\begin{center}
\textit{This page intentionally left (almost) blank.}\thispagestyle{empty}
\end{center}
\vspace{\fill}}

\makeatletter
% start with some helper code
% This is the vertical rule that is inserted
\newcommand*{\algrule}[1][\algorithmicindent]{\makebox[#1][l]{\hspace*{.5em}\vrule height .75\baselineskip depth .25\baselineskip}}%

\newcount\ALG@printindent@tempcnta
\def\ALG@printindent{%
    \ifnum \theALG@nested>0% is there anything to print
        \ifx\ALG@text\ALG@x@notext% is this an end group without any text?
            % do nothing
            \addvspace{-1pt}% FUDGE for cases where no text is shown, to make the rules line up
        \else
            \unskip
            % draw a rule for each indent level
            \ALG@printindent@tempcnta=1
            \loop
                \algrule[\csname ALG@ind@\the\ALG@printindent@tempcnta\endcsname]%
                \advance \ALG@printindent@tempcnta 1
            \ifnum \ALG@printindent@tempcnta<\numexpr\theALG@nested+1\relax% can't do <=, so add one to RHS and use < instead
            \repeat
        \fi
    \fi
    }%
% the following line injects our new indent handling code in place of the default spacing
\patchcmd{\ALG@doentity}{\noindent\hskip\ALG@tlm}{\ALG@printindent}{}{\errmessage{failed to patch}}
\makeatother
% end vertical rule patch for algorithmicx

\renewcommand{\algorithmicforall}{\textbf{for each}}

\def\mypart#1#2{%
\refstepcounter{part}% Next part
\addcontentsline{toc}{part}{\thepart~#1}
\par\newpage\clearpage % Page break
\vspace*{5cm} % Vertical shift
{\centering \textbf{\Huge Part \thepart}\par}\thispagestyle{empty}%
\vspace{1cm}% Vertical shift
{\centering \textbf{\Huge #1}\par}%
\vspace{2cm}% Vertical shift %
\noindent #2
\vfill\pagebreak % Fill the end of page and page break
}


%semantik commands
\newcommand{\Tfor}{\mathbin{\text{for}}}
\newcommand{\Tin}{\mathbin{\text{in}}}
\newcommand{\Twhile}{\mathbin{\text{while}}}
\newcommand{\Tx}{\mathbin{\; \text{x} \;}}
\newcommand{\Tlet}{\text{let}}
\newcommand{\Tvar}{\text{var}}
\newcommand{\Tspawn}{\mathbin{\text{spawn}}}
\newcommand{\Tsend}{\mathbin{\text{send}}}
\newcommand{\Ta}{\; \mathbin{\text{a}} \;}
\newcommand{\Tm}{\; \mathbin{\text{m}} \;}
\newcommand{\Twhere}{\mathbin{\text{where}}}
\newcommand{\Ten}{\mathbin{\text{env}}}
\newcommand{\Tactor}{\mathbin{\text{actor}} \;}
\newcommand{\Tact}{\; \mathbin{\text{act}} \;}
\newcommand{\Trecieve}{\mathbin{\text{recieve}} \;}
\newcommand{\Tr}{\; \mathbin{\text{r}} \;}