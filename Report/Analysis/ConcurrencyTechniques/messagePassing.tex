\section{Message Passing}

One paradigm of writing concurrent programs is message passing. The paradigm assumes a distributed process memory model in which each process has its own local adress space. The processes cooperates in solving a task by doing computations on their own local data and sending messages to each other. Each process can run independently on other processes as only the local data in the process can be accessed and modified.

Source: http://cds.cern.ch/record/399393/files/p165.pdf

\url{https://computing.llnl.gov/tutorials/parallel_comp/#ModelsShared}
