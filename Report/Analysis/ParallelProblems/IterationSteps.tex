\chapter{Models}
Model is a term often used in both natural and social sciences. Unfortunately the term is not as well defined as it perhaps should be, and it will therefore here be specified what is meant by it. The approach here is based on Mario Bunge definition of a model. According to his theory a model consists of two parts:
\begin{itemize}
\item A general theory
\item A special description of an object or system (model object)
\end{itemize}

Genrally there are two different types of models, static and dynamic models.
\subsubsection{Static models}
A static model is a snapshot of a given system at a given time. These systems can be 

Dynamic and static models


In this project a combination between static and dynamic modelling will be used in a [maybe] 

\chapter{Simulations}
% We make for ourselves internal images or symbols of the external objects, and we make them in such a way that the consequences of the images that are necessary in thought are always images of the consequences of the depicted objects that are necessary in nature ... Once we have succeeded in deriving from accumulated previous experience images with the required property, we can quickly develop from them, as if from models, the consequences that in the external world will occur only over an extended period or as a result of our own intervention.