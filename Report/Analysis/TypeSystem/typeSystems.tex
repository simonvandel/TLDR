\section{Type Systems}

The goal of the type system is to minimise programming errors, but on the other hand not being too restrictive so the expressiveness of the language is reduced. 

Languages can generally be categorised into dynamically typed and statically type languages. Statically typed languages have the following advantages over dynamically typed languages.

\begin{itemize}
  \item Type errors are presented at the soonest possible level: compile-time. This makes it impossible to have a program stop unexpectedly because of a type error at run-time
  \item Because of the fact that types are checked at compile time, no overhead is imposed to the compiled program at run-time
\end{itemize}

Of course, dynamically typed languages have its uses. One can argue that prototyping a program is done faster in a dynamically typed language. The trade-off is usually expressiveness for safety and performance.

In order to reduce the mental overhead and the number of explicit type declarations to write in a language, type inference can be implemented in the compiler. Having almost all types inferred by the compiler makes the language have more of the same feel as a dynamically typed language, in that there is less mental overhead for the user. Using a statically typed language with type inference brings all the beneficial properties of statically typed language, combined with a \enquote{lighter} feel.

\subsection{Strictness}

Type systems are often described as \emph{strong} or \emph{weak}. No formal definitions of such categorisations of type systems exist, so this report will use the following understanding of \emph{weak} and \emph{strong} type systems. A type system goes from \emph{weak} to \emph{strong} as the number of undefined behaviour, unpredictable behaviour or implicit conversions between types approach zero.

\subsection{Higher Kinded Types}
\label{sub:higher_kinded_types}

Some programming languages allow the notion of higher kinded types, i.e. types of types. A kind is a type of a type constructor. One can describe a kind as one level up from a type. Kinds are often notated using \emph{*}. For example the kind of the type \emph{int} is \emph{*}. The kind of the function type \emph{int -> int} is of kind \emph{* -> *}.

Type systems with support for higher kinded types allow for even more guarantees that can be made about the program at compile-time.
