\section{Type Systems}

The goal of the type system is to minimise programming errors, but on the other hand not being too restrictive so the expressiveness of the language is reduced. 

\subsection{Strictness}
Type systems are often described as \emph{strong} or \emph{weak}. No formal definitions of such categorisations of type systems exist, so this report will use the following understanding of \emph{weak} and \emph{strong} type systems. A type system goes from \emph{weak} to \emph{strong} as the number of undefined behaviour, unpredictable behaviour or implicit conversions between types approach zero.

\subsection{Static versus dynamic type systems}
Languages can generally be categorised into dynamically typed and statically type languages. The difference between static and dynamic typing is at the time the types are checked. In static typing, types are checked at compile time. In dynamic typing, types are checked at run-time.

Statically typed languages have the following advantages over dynamically typed languages.

\begin{itemize}
  \item Type errors are presented at the soonest possible level: compile-time. This makes it impossible to have a program stop unexpectedly because of a type error at run-time
  \item Because of the fact that types are checked at compile time, no overhead is imposed to the compiled program at run-time
\end{itemize}

Of course dynamically typed languages have its uses. One can argue that prototyping a program is done faster in a dynamically typed language because the programmer does not need to think of types when programming. The mental overhead is usually lower. The trade-off of dynamic typing versus static typing is often expressiveness for safety and performance.

\subsection{Type Inference}
Type checking solves the problem of determining if a program is well-typed i.e. does not have any type errors. Type inference can be seen as solving the opposite problem; determine types in a program that makes the program well-typed.

In order to reduce the mental overhead and the number of explicit type declarations to write in a language, type inference can be implemented in the compiler. Having almost all types inferred by the compiler makes the language have more of the same feel as a dynamically typed language, in that there is less mental overhead for the user. Using a statically typed language with type inference brings all the beneficial properties of statically typed language, combined with a \enquote{lighter} feel.

\subsection{Summary}
Type systems are defined by the type rules it enforces. Because the strictness of the type rules can vary, type systems can range from being very strict to being very weak.

Generally two different approaches exist to solve the goal of minimising errors in programs. Static typing checks types at compile-time and therefore is able to present type errors at a early stage. Because of this, static typing is often regarded as safer and more performant.

Dynamic typing checks types at run-time and therefore allows the programmer to not have the same mental overhead of thinking about types when programming. This however comes at the price of safety and performance.

Type inference is a solution to having to explicitly annotate programs with type annotations. This can reduce the gap of mental overhead between dynamic typing and static typing.


