\chapter{Problem Domain}\label{part:analysis}

In this chapter, the problem domain will be explained and analysed, in order to provide a better understanding of the intentions with The Language Described in this Report, and the reasoning upon which choices and delimitations will be made. The chapter will briefly cover reasons for modelling and different approaches to modelling a real world scenario.
Following this is a section on simulations in general, and specific ways of doing theoretical simulations. This also addresses some problems with the creation of computer simulations, and how such a problem can be viewed.

\section{Simulation}

In this project the scope of parralisation is within simulations, this section will describe some general properties and processes of computer simulations. At last the section will discuss generalisations and assumpstions that can be made on the basis of the understading in this problem domain, which can be used to provide the design phase with some ideas of abstractions the language should implement.

The main benefit of parallelising anything is to gain \emph{speed up} in the computation time of problems, this will be described in \cref{sup}. This is done by running multiple tasks within a problem, at the same time(in parallel) or concurrently in different threads of multicore system.

\emph{Understanding the problem domain}
Firstly before spending time in an attempt to develop a parallel solution for any problem, one should first determine whether or not the problem is one that can actually be parallelized. This will be described in \cref{top}.

\emph{Granuality}
  In order to do a computable simulation of a problem, a level of granuality is to be determined, this is a significant for both computation time and accuracy of the simuilation. A description of this process is giving in \cref{dis}

\emph{Helping processes and tools in making solutions parallel}
    Designing and developing parallel programs is characteristically a very manual process. The programmer is typically responsible for both identifying and actually implementing parallelism.

    Manually developing parallel solutions is a time consuming, complex, error-prone and an iterative process.

    Various tools can assist the programmer with converting serial programs into parallel programs. The most common type of tool used to automatically parallelize a serial program is a parallelizing compiler or pre-processor.

    This is usually a process of the compiler analysing the source code of a serial solution and identifying opportunities for parallelism.
    The analysis includes identifying inhibitors to parallelism and possibly a cost weighting on whether or not the parallelism would actually improve performance.
    For example loops (do, for) are the most frequent subject to automatic parallelization.


\chapter{Conditions for a Simulation Language}\label{part:analysis2}

In this chapter, we will examine certain technological tools for handling problems within the domain of computer simulations. This will include a section on high performance computing (HPC), which will cover the systems that typically compute simulations. Right after the HPC section will be a section on parallelism, based on theory from computer science. This section on parallelism covers the attractiveness of parallelism, as well as the problems which can occur in a parallel system. Then a section on the different memory architectures used in parallelism, and a section on the usage of the Message Passing Interface (MPI).

This all precedes a section on the actor model, and how this model can be used to do simulations, and where problems can arise with it.

\chapter{High Performance Computing}
High Performance Computing (from here on, HPC) is a field of computing, where the main objective is performance, often measured in floating point operations per second (FLOPS). The term HPC is typically associated with supercomputers and large distributed system.

\section{Use}
HPC is rarely used by your "average Joe", but more so by scientists and engineers who need to perform computation on very large datasets or "big data"', model reality with a lot of detail or calculate very large equations.

\section{Supercomputers and Distributed Systems}
Nearly all platforms for HPC are either supercomputers or distributed systems. Contemporary supercomputers are getting more and more powerful, with the current world leader, the Chinese Tianhe-2, having a theoretical peak performance of almost 55 petaFLOPS [http://www.top500.org/system/177999]. 
Distributed systems generally consist of a large group, or "cluster", of computers, referred to as "nodes". In these systems, the individual nodes might not be more powerful than an ordinary desktop PC, but the collective processing power of the whole cluster can rival that of a supercomputer. 
A simple cluster could consist of the following nodes:
\begin{itemize}
	\item Head
	The controller of the cluster. This is where the cluster's interface is, and where the jobs that should be performed are first introduced to the system.
	\item Compute
	A generic worker. A compute node receives a job and returns the result of the computation. In a normal cluster, you would have many compute nodes.
	\item Scheduler
	A node with the sole purpose of sending jobs to compute nodes. A good scheduler will evenly distribute the load of jobs to the compute nodes, minimising the total computation time.
\end{itemize}

These systems' performance is so high mainly due to their extensive use of parallel computing. This parallelism is achieved, in the case of Tianhe-2, by using MPICH2, one of the most popular implementations of the MPI specification for message passing interfaces.

\section{Current Technology}
Even though the hardware used in HPC-systems evolves at a very high rate, unfortunately the same cannot be said for the programming languages used to program the systems. The programming languages almost solely used to program HPC-systems are C and Fortran. At the time of writing, C is 43 years old and Fortran is 58 years old. For any technology related to software to survive this long is quite a feat, but also rather worrying considering its heavy use in HPC, what should be the milestone for efficient programming. 
So why is such a critical performance-oriented field still dominated by technologies which are about half a century old? There are three very big factors in why these languages are dominant:
\begin{enumerate}
	\item Tried and true
	Both C and Fortran have existed for such a long time, that both have had the time needed to mature. Every feature in the languages is almost guaranteed to be correct, as the languages have been extensively used and developed for such a long time.
	\item Support
	Nearly all general computational problems have been solved in C and Fortran, meaning that for almost all trivial problems, someone else has already done all the work and you can merely use their code.
	\item Near to the metal
	As C and Fortran are rather low level languages, there is a large amount of control in the hands of the developer. This means that there is a lot of potential for optimisation in the code.
\end{enumerate}

OpenMP/MPI?\\
Nye HPC sprog?
\begin{itemize}
	\item Fortress (Sun)
	\item Chapel (Cray)
	\item X10 (IBM)
\end{itemize}
%Parallelitet:
%- Hvorfor lave noget parallelt?
%- Hvorfor er parallelitet fremtiden?
%  - Fysisk grænse for hvor hurtigt en processor kan køre
%Hvorfor sprog til det (merge måske med intro):
%- Flere og flere får adgang til clusters længere "nede" og man kan bygge små distribueret systemer lokalt (processor kort til din stationære)
%  - Snak om at det derfor skal være simpelt


\subsection{Parallelism}\cite{parallel programming languages}.

In the continued effort of trying to squeeze as much power out of computers as possible, the computer scientific community is at a point where increasing the clock speed of processors is no longer as viable a solution as it used to be. This has spawned an increased interest in increasing computational power in other ways, one being parallelism.

Parallelism is the act of dividing calculations into independently solvable parts, and then solving them on multiple processors before finally gathering the individual results and combining them. The main benefit of parallelising anything is to gain \emph{speed up} in the computation time of problems. This will be described in \cref{sup}.

With parallelism you gain \emph{speed up} through combining multiple processing units. This is seen in newer CPU's as multiple cores, but on a larger scale this principle can be used to create supercomputers, capable of performing immense calculations.

But even without a supercomputer a distributed network of multiple computers can provide with large amounts of parallel computing power. With this being a foreseeable future, we predict an increase in the access to, and need for, parallel systems.
\chapter{Parallel Computer Memory Architectures}

\section{Shared Memory}

All processors in the computer have access to the same memory, that is, the memory address space is global. The memory is in close proximity to the processors, providing lower latency of access. The processors can operate independently of each other, but share the same memory resources. Because of this sharing, a processor A can affect memory used by processor B. This leads to non-deterministic programs where a processor can not be sure what state the memory is in at a given time, without any syncronisation between processors.

\section{Distributed Memory}

Processors have their own local memory. It is not possible to directly access memory between processors. That is, each processor has their own \enquote{view} of the memory address space. Therefore, processors cannot affect each others memory. For two processors to share data, the data must be sent via a data transfer link. Depending of the proximity between the two processors, the speed and latency may be subpar compared to the shared memory architecture described above.

\section{Architecture Choice}

The programming language defined in this report is targeted at scientists wanting to perform simulations. Because of the need to perform these simulations quickly, even without super-computers, there should be support for connecting several computers in a network to create a distributed system. The distributed memory architecture is therefore chosen as the prefered architecture choice for this programming language.
\chapter{Message-Passing Interface Standard}
Message-Passing Interface Standard (MPI) the de facto standard for message passing when doing high performance computation. It was first released in 1994 as version 1.0. It were created in collaboration between 40 different organisations involving about 60 people. It was based in a preliminary proposal, known as MPI1, that was created by Jack Dongarra, Rolf Hempel, Tony Hey, and David Walker and finished in November 1992 and inspired from current practice, such as PVM, NX, Express, p4, etc. It has since been iterated upon, with the latest version, the MPI-3.0, comming out in 2012.

The reason for creating this standard is to ensure portability and ease-of-use. In a distributed memory communication environment high-level routines and abstractions are often build upon lower level message passing routines of which this standardisation ensures that these abstractions and routines are available on all systems. This also present vendors with few clearly defined base routines they need to implement efficiently and, if possible, provide hardware support for enhancing scalability.

MPI is designed to allow efficient communication, use in heterogeneous environments and possibility for thread-safety. The semantics of the interface itself should be independent from any language it is implemented in, however it does focus on providing convenient bindings for the C and Fortran languages. MPI also specifies how communication failures are dealt with in the underlying systems.

The standard is designed for message passing only and does not include specifications for explicit operations on shared-memory. Operations that goes beyond the current operating system support are not included, such as interrupt-driven receives, remote execution, or active messages, since the interface is not intended for kernel programming/embedded systems. Neither does it include support for debugging, explicit support for threads or task management or support for I/O functions.

(Source: http://www.mpi-forum.org/docs/mpi-1.0/mpi-10.ps chapter 1)

\section{Messagetypes}

\subsection{Point-to-point operations}
\subsubsection{Send}
Point-to-point communication is the the basic building block of message passing. Here one process sends data in form of a message to a single other process.
\subsubsection{Recieve}
When a process receives the message enqueues the message in a queue called its message box. Each message in the message box is processed sequentially by dequeuing and handling them one at the time.

(Source: http://www.mpi-forum.org/docs/mpi-1.0/mpi-10.ps chapter 3)

\subsection{Collective operations}
\subsubsection{Broadcast}
Broadcast is a one-to-many operation, where one one process has some specific data that it sends to many other processes. The data is therefore multiplied, as opposed to being divided.

\subsubsection{Scatter}
Scatter is a close relative to broadcast. It is also a one-to-many operation but here the data is divided into equally large chunks and is distributed among multiple processes (including the process originally containing all the data). This could for instance be sending all entries in a collection to different processes that individually process their part.

\subsubsection{Gather}
Gather is a many-to-one operation and is the inverse of scatter. Here data from many processes are sent to one of them. This operation often implies a hierarchy of the processes containing the data where the process highest in the hierarchy receives all the data (also from itself).

\subsubsection{Reduce}
Reduce is a many-to-one operation. Here one operation, called the reduction function, is done on data from multiple processes and the result is placed in one of them. As in gather a hierarchy is also custom and the process highest in the hierarchy receives the result of the reduction. All reduction functions must be both associative and commutative so results can be reduced asymmetrically as data is sent from processes.
(Source: http://www.mpi-forum.org/docs/mpi-1.0/mpi-10.ps chapter 4)

\subsection{System buffer}
Consider the following; When sending messages, what happens if the receiver is not ready to process them? To solve this problem, MPI dictates that an implementation must be able to store messages, however the way this is done is up to the implementation.

One way to do this is with a system buffer. In short terms a system buffer works as an inbox, and sometimes also an outbox, where messages are stored until they can be processed. A few things to note about this inbox, is that it is supposed to work behind the scenes, not manageable by the programmer. However, what the programmer do need to realize, is that this buffer is a finite resource, which will be exhausted if one is not cautious.
(Source: MPI\_tut1)

\subsection{Blocking and Non-blocking sends}
On messages there can be made two distinct groups, the blocking sends and the non-blocking sends. The straight forward approach is the non-blocking send, where the sender assumes or is certain that the receiver is ready to handle the message. These types of messages returns almost immediately, but there is no guarantee that the message was received, or how it was received.

On the other hand there is the blocking send. This only returns after it is safe to modify the application buffer again. The sent data could still be sitting in a buffer, but it is considered safe.

Generally blocking sends are used for operations, and non-blocking when you wish to overlap those with communication in order to improve performance.
(Source: MPI\_tut2)

\subsection{Order and Fairness}
When sending messages, the order in which they are sent and receive can matter a great deal. MPI gives a few guarantees as to how this can be expected to happen. One must note that these rules are not enforced if multiple threads are participating in the communication.

When using MPI, messages shall not overtake each other. If one task sends out two messages in succession, the message sent out first will be the first one to be received. Also, if a receiver have multiple requests looking for a matching message, the request that was made first, will get the first match.

MPI does not give any guarantees related to fairness. It is entirely possible to starve tasks. If this is not desired, the responsibility lies with the programmer.
(Source: MPI\_tut3)
\section{The Actor Model}
The actor model is a concurrency model, which first appeared in 1973 in the report "A Universal Modular ACTOR Formalism for Artificial Intelligence" by Carl Hewitt, Peter Bishop and Richard Steiger [worrydream.com/refs/Hewitt-ActorModel.pdf]. This chapter will explain the model generally. In addition to the general model, the model will be compared to alternative concurrency models and a few implementations of the model will be briefly described.

\subsection{Model}
The actor model is focused on the fact that any computational behaviour, be it functions, processes or data structures, can be modelled as a single behaviour; sending messages to actors.
An actor in this context is a process, which has its own isolated state, used to store values and modify its own behaviour, and a message box, for receiving messages. An actor can only share information by sending messages, asynchronously, to other actors. 
Actors are very similar to the communicating sequential processes (CSP) described in C.A.R. Hoare's publication "Communicating Sequential Processes". Actors, though, have a few well-defined behaviours in response to receiving a message, where CSPs are free to exhibit other behaviour. An actor can, in response to a message: 
\begin{enumerate}
  \item Send a finite number of messages to other actors.
  \item Spawn a finite number of new actors.
  \item Change its own behaviour for the next message that is received.
\end{enumerate}

In figure \ref{fig:actor}, a model of an actor based program can be seen.

\begin{figure}
  \includegraphics[width=\textwidth]{Images/actors.pdf}
  \caption{A model of an example program using actors. The nodes in this graph represent actors and the edges represent messages being sent.}
  \label{fig:actor}
\end{figure}


\subsubsection{Adding Logic}
One of the main features of the actor model is, as Hoare describes it, the ability to add knowledge to a system, without rewriting the existing knowledge. This means that if a developer is interested in adding a logical step in the business logic of his system, this can be achieved by merely adding an actor, which can perform the logical operations, that are to be added, and then adding that actor as a link in the flow of logic.

\paragraph{Example}
We have a system which can read a CSV-file and print it to stdout. This can be implemented using actors in the following way:\\
let Reader and Printer be actors.

\begin{tabular}{ | c | c | c | }
\hline
Step & Reader & Printer \\\hline
1 & \textbf{Read CSV-file} & \textit{Wait for message} \\\hline
2 & \textbf{Send data to Printer} & \textit{Wait for message}\\\hline
3 & \textit{Wait for message} & \textbf{Receive data from Reader}\\\hline
4 & \textit{Wait for message} & \textbf{Print data to stdout}\\\hline
\end{tabular}\\
At som point we decide that we wish to format our data before printing it. To do so, we create and incorporate a new actor, Formatter, which changes the flow of execution in this manner:

\begin{tabular}{ | c | c | c | c | }
\hline
Step & Reader & Formatter & Printer \\\hline
1 & \textbf{Read CSV-file} & \textit{Wait for message} & \textit{Wait for message} \\\hline
2 & \textbf{Send data to Formatter} & \textit{Wait for message} & \textit{Wait for message}\\\hline
3 & \textit{Wait for message} & \textbf{Receive data} & \textit{Wait for message} \\\hline
4 & \textit{Wait for message} & \textbf{Format data} & \textit{Wait for message} \\\hline
5 & \textit{Wait for message} & \textbf{Send data to Printer} & \textit{Wait for message} \\\hline
6 & \textit{Wait for message} & \textit{Wait for message} & \textbf{Receive data}\\\hline
7 & \textit{Wait for message} & \textit{Wait for message} & \textbf{Print data to stdout} \\\hline
\end{tabular}\\

Notice that in adding the new logic, the only existing logic that was changed is the actor whom Reader sends data to.

\subsubsection{Supervisors}
Most implementations of the actor model implement a form of a supervisor-worker relationship. This means that every actor has a supervisor, typically their parent, to whom they report failures. The supervisor then has to deal with the failure, by for example restarting the failed actor.
Supervisors in the actor model works very well with the "fail fast"-principle of programming, meaning, in the actor model, that an actor will not try to continue operation or correct an error when encountering a failure. An actor that has failed will, as soon as the failure is detected, report the failure to its supervisor and halt operation, moving the responsibility of handling the failure up the supervision tree.

\subsection{Implementations}
\subsubsection{Erlang}
When talking about the actor model, one cannot escape the subject of Erlang. Erlang is one of first languages to fully incorporate the actor model as the main model of concurrency. Erlang is a purely functional language, developed for the telecommunications industry, where a lot of concurrent processes have to be handled. 
Actors in Erlang, or processes as they are called, are controlled by a few very simple constructs; actors can be spawned using the built-in spawn-function, the spawn-function takes three arguments, the module in which the actor is contained, the function that defines the behaviour for the actor and an initial message. You can send messages to actors using the bang-operator (!) and you can define an actor's behaviour in a receive-block.
In Erlang, an actor's parent is it's supervisor, if one chooses to use the supervisor module available. Using the supervisor module in Erlang gives the programmer the ability to choose different strategies to be used when a child actor fails. These strategies are:
\begin{enumerate}
  \item Respawn the failed child
  \item Respawn all children when one fails
  \item Respawn all children after the child in the start order of the children
\end{enumerate}

\paragraph{Using actors in Erlang - Example}
To demonstrate the use of actors in Erlang, a simple example is shown in listing \ref{lst:ErlExample}. This program has only one actor, which counts the number of "incr"-messages it has been sent.

\begin{lstlisting}[style=erlang, caption={A simple message-counter in Erlang.}, label=lst:ErlExample]
-module(countMsgs).
-export([run/0, counter/1]).

run() ->
  S = spawn(countMsgs, counter, [0]), %spawn S as counter-actor
  sendMsgs(S, 10000), %send 10000 messages to S
  S.
  
counter(Sum) -> %function-definition for counter-actor
  receive
    value -> io:fwrite("Value is ~w~n", [Sum]);
    incr -> counter(Sum+1)
  end.

sendMsgs(_, 0) -> true; %base case
sendMsgs(S, N) -> %recursive function to send N messages to actor S
  S ! incr, %send incr to S
  sendMsgs(S, N-1).
\end{lstlisting}

\subsubsection{Akka}
A more modern approach to the actor model is taken in Akka, a concurrency toolkit for Scala and Java.
In Akka, an actor is just an object, which extends Akka's Actor-class and implements a receive method. One difference in the receive-method from Erlang's receive-block is that Akka's receive-method is exhaustive, meaning that the programmer has to define behaviour for all possible messages that an actor can receive. 
Creating an actor in Akka is done by calling the actorOf-method, on an actor or an instance of ActorSystem, which acts as the parent of top-level actors. The actorOf-method takes a Probs as an argument, Probs is just an object containing properties for an actor, such as the definition of the actor. Just like Erlang, messages are sent using the bang-operator.
In Akka, supervisors have the ability to handle failures in child actors in the following manner:
\begin{enumerate}
  \item Ignore failure and resume operation
  \item Restart the child from initial state
  \item Terminate the child, without respawning it
  \item Fail (move failure up supervision tree)
\end{enumerate}

\paragraph{Using Actors in Akka - Example}
To demonstrate Akka's implementation of actors, the same example shown in listing \ref{lst:ErlExample} will be implemented in Scala, using Akka, in listing \ref{lst:AkkExample}.


\begin{lstlisting}[style = scala, caption={A simple message-counter in Scala.}, label=lst:AkkExample]

import akka.actor.Actor
import akka.actor.Probs

class Counter extends Actor{ //definition of counter-actor
  var count = 0
  def receive = {
    case "incr" => count += 1 //if actor receives "incr", increment count
    case "value" => println(s"Value is $count")
    case other => println(s"Error: Cannot understand message $other") //default case
  }
}

object Main extends App {
  val mySystem = ActorSystem("MySystem")
  val myCounter = mySystem.actorOf(Probs[Counter], name="MyCounter") //create counter-actor as top-level actor
  def main(args:Array[String]){
    for (i <- 1 to 10000) { //send 10000 messages to myCounter
      myCounter ! "incr"
    }
  }
}
\end{lstlisting}


\section{Criteria for Language Design}\label{analsum}

This section will, in light of the preceeding two chapters, \cref{part:analysis1} and \cref{part:analysis2}, settle on some criteria for The Language Described in this Report. The criteria are inspired by the ones found in \kanote{indsæt ref til sebesta bog}. However, only those characteristics which differentiates the language will be discussed, even if it might be present in the language.


\subsection{Simplicity, orthogonality and Syntax Design}

In order to accommodate a usergroup with programming as secondairy skill sets, The Language Described in this Report should be simple. This would result in strict and straitforward rules regarding interactions. The language should try to keep simple rules regarding orthogonality as well, but not necessarily be highly orthogonal for that reason. One such rule, based on properties of the actor model, is that actors should, in the eyes of the programmer, only communicate through messages either sent or received. This would prohibit directly adressing fields in an actor, which would lower orthogonality, but keep the language simple. The syntax design, should for the same reasons, caters towards a modelling perspective and not nessesarily a technical accurate perspective.


\subsection{Data Types}

Since the users of the language are usually either with background in mathematics, or require to include some mathematics in order to asses the value of any given result of a simulation. Therefore the language should strive towards having data types which allow for traditional mathematical numbers.


\subsection{Type checking and Exception Handling}

Due to the language targeting high performance computation, the language should try to avoid runtime errors, and catch problems as early as possible. This suggests a strongly typed language, but we will elaborate further on this matter in \cref{typesys}. For the same reasons, the language will not value exception handling, since edge cases should be fully incompassed in a simulation. However it is worth noticing that the actor model could require exception handling, if communication problems, caused by either race conditions or network conditions, should prove frequent.


%\subsection{Support for abstraction}


%\subsection{Aliasing}
